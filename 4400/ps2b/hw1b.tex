\documentclass[fleqn]{article}

\usepackage{mydefs}
\usepackage{notes}
\usepackage{url}

\begin{document}
\lecture{CS4400}{Lecture 4 Problem Set}{Alex Clemmer, u0458675}

\section*{Problem 2.86}

\bee
\i Give a correct implementation using only left and right shifts along with 1 subtraction.

 \begin{solution}
 	\begin{tabular}{ l c r }
  		1 & 2 & 3 \\
  		4 & 5 & 6 \\
  		7 & 8 & 9 \\
	\end{tabular}
 \end{solution}

\ene

\section*{Problem 2.88}

\bee
\i \texttt{(float) x == (float) dx}

% ANY LINE BEGINNING "%" IS A COMMENT.  YOU CAN UNCOMMENT THE BELOW
% TEXT AND FILL IN YOUR OWN.
 \begin{solution}
 True. Rounding may cause problems in the general case, but \texttt{float} is well-defined for the range this deals with. (I verified this experimentally.)
 \end{solution}

\i \texttt{dx - dy == (double) (x - y)}

% ANY LINE BEGINNING "%" IS A COMMENT.  YOU CAN UNCOMMENT THE BELOW
% TEXT AND FILL IN YOUR OWN.
 \begin{solution}
 True. \texttt{double} is well-defined over the range of \texttt{int}, so there is no chance of a rounding error. We're dealing in integers, so decimal rounding errors are out. There is also no way to cause ambiguous 0 comparisons.
 \end{solution}

\i \texttt{(dx + dy) + dz == dx + (dy + dz)}

% ANY LINE BEGINNING "%" IS A COMMENT.  YOU CAN UNCOMMENT THE BELOW
% TEXT AND FILL IN YOUR OWN.
 \begin{solution}
 True. Not true in the general case, as floating-point arithmetic is not associative or distributive, but \texttt{double} encapsulates all numbers possible in \texttt{int}, and therefore is well defined for these conditions.
 \end{solution}

\i \texttt{(dx * dy) * dz == dx * (dy * dz)}

% ANY LINE BEGINNING "%" IS A COMMENT.  YOU CAN UNCOMMENT THE BELOW
% TEXT AND FILL IN YOUR OWN.
 \begin{solution}
   False. Distributed 
 \end{solution}

\i \texttt{dx / dx == dz / dz}

% ANY LINE BEGINNING "%" IS A COMMENT.  YOU CAN UNCOMMENT THE BELOW
% TEXT AND FILL IN YOUR OWN.
 \begin{solution}
   False. If \texttt{dx} or \texttt{dz} is \texttt{0} while the other isn't, then they will have different results.
 \end{solution}

\ene

\end{document}
