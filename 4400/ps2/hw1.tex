\documentclass[fleqn]{article}

\usepackage{mydefs}
\usepackage{notes}
\usepackage{url}

\begin{document}
\lecture{CS4400}{Lecture 3 Problem Set}{Alex Clemmer, u0458675}

\section*{Problem 2.71}

\bee
\i What is wrong with this code?

% ANY LINE BEGINNING "%" IS A COMMENT.  YOU CAN UNCOMMENT THE BELOW
% TEXT AND FILL IN YOUR OWN.
 \begin{solution}
   This function guarantees in a lot of cases that the sign extension is not kept when we mask with 0xFF.
 \end{solution}

\i Give a correct implementation using only left and right shifts along with 1 subtraction.

 \begin{solution}
 	\begin{verbatim}
 	int xbyte(packed_t word, int bytenum)
 	{
        /* Shift all the way to the right */
        int t = (3 - bytenum) << 3;
 			
        /* Shift back all the way to the left */
        int32_t res = (word << ti) >> 24;
 	}
 	\end{verbatim}
 \end{solution}

\ene

\section*{Problem 2.76}

NOTE: I'll assume that the variable we're multiplying is called \texttt{x}.

\bee
\i K = 17

% ANY LINE BEGINNING "%" IS A COMMENT.  YOU CAN UNCOMMENT THE BELOW
% TEXT AND FILL IN YOUR OWN.
 \begin{solution}
   \texttt{x = (x << 4) + x}
 \end{solution}

\i K = -7

% ANY LINE BEGINNING "%" IS A COMMENT.  YOU CAN UNCOMMENT THE BELOW
% TEXT AND FILL IN YOUR OWN.
 \begin{solution}
   \texttt{x = x - (x << 3)}
 \end{solution}

\i K = 60

% ANY LINE BEGINNING "%" IS A COMMENT.  YOU CAN UNCOMMENT THE BELOW
% TEXT AND FILL IN YOUR OWN.
 \begin{solution}
   \texttt{x = (x << 6) - (x << 2)}
 \end{solution}

\i K = -112

% ANY LINE BEGINNING "%" IS A COMMENT.  YOU CAN UNCOMMENT THE BELOW
% TEXT AND FILL IN YOUR OWN.
 \begin{solution}
   \texttt{x = (x << 4) - (x << 7)}
 \end{solution}

\ene

\section*{Problem 2.81}

\bee
\i \texttt{(x<y) == (-x>-y)}

% ANY LINE BEGINNING "%" IS A COMMENT.  YOU CAN UNCOMMENT THE BELOW
% TEXT AND FILL IN YOUR OWN.
 \begin{solution}
   Signed range is asymmetric, so supplying \texttt{x = INT\_MIN} (or equivalent minimum value) and almost anything for \texttt{y} will probably break this system, as negating \texttt{INT\_MIN} gives us \texttt{INT\_MIN}, and nothing is strictly smaller than \texttt{INT\_MIN}, which is required for the righthand expression to be true.
 \end{solution}

\i \texttt{((x+y)<<4) + y-x == 17*y+15*x}

% ANY LINE BEGINNING "%" IS A COMMENT.  YOU CAN UNCOMMENT THE BELOW
% TEXT AND FILL IN YOUR OWN.
 \begin{solution}
   True. \texttt{16*(x+y)+y-x == 16x-x+16y+y == 17y+15x}. There are no corner cases like there were in the last one.
 \end{solution}

\i \texttt{$\sim$x+$\sim$y+1 == $\sim$(x+y)}

% ANY LINE BEGINNING "%" IS A COMMENT.  YOU CAN UNCOMMENT THE BELOW
% TEXT AND FILL IN YOUR OWN.
 \begin{solution}
   True. First, \texttt{$\sim$x+$\sim$y+1 == -x-1+($\sim$y) == -x-1-y}. Then, \texttt{$\sim$(x+y) == -(x+y)-1 == -x-y-1}. So they are equivalent. 
 \end{solution}

\i \texttt{(ux-uy) == -(unsigned)(y-x)}

% ANY LINE BEGINNING "%" IS A COMMENT.  YOU CAN UNCOMMENT THE BELOW
% TEXT AND FILL IN YOUR OWN.
 \begin{solution}
   False. If \texttt{x = -10} and \texttt{y = -1}, then \texttt{10-1 == 9} and \texttt{-(unsigned)(-10-(-1)) == }
 \end{solution}
 
 \i \texttt{((x >> 2) << 2) <= x}

% ANY LINE BEGINNING "%" IS A COMMENT.  YOU CAN UNCOMMENT THE BELOW
% TEXT AND FILL IN YOUR OWN.
 \begin{solution}
   True. The binary representation of a number increases monotonically as the number itself increases even in the case of negative numbers (\textit{i.e.}, for any number $n$, the binary representation of $n+1$ is larger than $n$ was) and therefore when you divide by two and multiply by a power of 2 (\textit{e.g.}, $x = x/n * 2^n$), you will at best end up with the same number, and at worst, a smaller number.
 \end{solution}

\ene

\end{document}
