\documentclass[fleqn]{article}

\usepackage{mydefs}
\usepackage{notes}
\usepackage{url}

\begin{document}
\lecture{CS4400}{Lecture 5 Problem Set}{Alex Clemmer, u0458675}

\section*{Problem 1}

Match each of the three IA32 assembly-code routines on the left with the equivalent C function on the right.

% ANY LINE BEGINNING "%" IS A COMMENT.  YOU CAN UNCOMMENT THE BELOW
% TEXT AND FILL IN YOUR OWN.
 \begin{solution}
   \texttt{bar1} matches to \texttt{foo6}. \texttt{bar2} matches to \texttt{foo1}. \texttt{bar3} matches to \texttt{foo5}.
 \end{solution}

\section*{Problem 3.55}

 \begin{solution}
 \begin{verbatim}
 int foo(int x, y, z)
 {
     int ret;
     y = y - z;
     ret = y;
     ret = ret << 31;
     ret = ret >> 31;
     y = y * x;
     ret = ret ^ y;
     
     return ret;
 }
 \end{verbatim}
 \end{solution}
 
 \section*{Problem 3}
 
 \begin{solution}
 \begin{verbatim}
 int foo(int *ptr, int a, short b, char c)
 {
     *ptr += a >> c;
     return -*ptr & b;
 }
 \end{verbatim}
 
 translates to:
 
 \begin{verbatim}
  ptr at %ebp+8, a at %ebp+12, b at %ebp+16, c at %ebp+20
 foo:
     pushl %ebp
     movl %esp, %ebp
     
     movl 12(%ebp), %eax   ; put a in register %eax
     sarl 20(%ebp), %eax   ; shift a arithmetic right by c
     movl 8(%ebp), %ecx    ; put address ptr in %ecx
     movl (%ecx), %edx     ; put value at ptr in %edx
     addl %eax, %edx       ; add ptr and the shifted amount
     movl %edx, (%ecx)     ; write value change to ptr
     negl %edx             ; negate ptr
     movswl 16(%ebp), %eax ; pull b out of memory
     andl %edx, %eax       ; and ptr and b; return
     
     leave
     ret
 \end{verbatim}
 
 \end{solution}

\end{document}
