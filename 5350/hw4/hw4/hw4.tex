\documentclass[fleqn]{article}

\usepackage{mydefs}
\usepackage{notes}
\usepackage{url}
\usepackage{verbatim}
\usepackage{fancyvrb}
\usepackage{color}

\definecolor{lightgreen}{rgb}{0.8,1.0,0.8}

\DefineVerbatimEnvironment%
  {matlab}{Verbatim}
  {baselinestretch=1.0,frame=single,fillcolor=\color{lightgreen}}

\begin{document}
\lecture{CS5350: Machine Learning}{HW4: Dimensionality Reduction, Ensemble 
Methods, Multiclass/Ranking}{Due November 17, 2011}

\section{Written Exercises}

\bee

\i Linear PCA does an eigen-decomposition of the $D\times D$ covariance
matrix of the data where $D$ is the original dimensionality of the data.
Kernel PCA (KPCA) which learns nonlinear projections is also based on doing an
eigen-decomposition of the covariance matrix but this time the covariance
matrix is defined in the feature space ($\phi$) defined by the kernel. This 
feature space could even be infinite dimensional (e.g., if you use an 
RBF/Gaussian kernel). Explain how does KPCA get away with \emph{not} computing
this covariance matrix and doing its eigen-decomposition? What does it
actually compute and does eigen-decomposition of? (\emph{10 points})

\i Dimensionality reduction methods take as input a set of points
$\{\vec{x}_1,\vec{x}_2,\ldots,\vec{x}_N\}$ and learn a low-dimensional
embedding (projection) of these points. Now suppose I give you a new point
$\vec{x}_{N+1}$. Of the various dimensionality reduction methods we 
have seen in the class (PCA, Kernel PCA, LLE, ISOMAP), which ones 
allow you to compute the low-dimensional projection of the new point
$\vec{x}_{N+1}$ (for each, explain why or why not). This problem is known
as the \emph{out-of-sample problem}. Note that including $\vec{x}_{N+1}$ in the original
data and re-running dimensionality reduction on the whole data is not an option.
(\emph{10 points})

\i For multiclass classification, All-vs-All (AVA) seems more computationally 
intensive at training time than One-vs-All (OVA) because it trains $\cO(K^2)$ 
classifiers rather than $\cO(K)$ classifiers.  However, all of the $K$-many OVA
classifiers are on the full data set of $N$ examples, while the
$\cO(K^2)$ AVA classifiers are only on subsets of the data.  Suppose
that you have $N$ data points, divided evenly into $K$ classes (so
that there are $N/K$ examples per class).

\begin{enumerate}
\item Suppose that the training time for your binary classifier is
  linear in the number of examples it receives.  What is the
  complexity of training OVA and AVA, as a function of $N$ and $K$?

%\begin{solution}
%  TODO
%\end{solution}

\item Suppose the training time is quadratic; then what is the
  complexity of AVA and OVA?


%\begin{solution}
%  TODO
%\end{solution}

 (\emph{10 points})

\end{enumerate}

\i One issue with multiclass classification based methods such as OVA
and AVA is that the test time complexity grows 
linearly or quadratically with $K$, the total number of classes. Moreover
the multiclass classification error also worsens as $K$ grows (because
basically we are using a collection of binary classifiers and their individual
errors would just add up). 

Now suppose we still want to construct a multiclass classification algorithm
based on a set of binary classifiers but want to control the error such that
it doesn't get worse than $\cO(\log_2 K)$. How would you accomplish this (hint:
what you want is minimizing the number of decisions you have to make at test time)?
(\emph{10 points})

\i Define a ranking preference function $\omega$ that penalizes
mispredictions \emph{linearly} up to a threshold $K$.  In other
words, for $K=20$, if I put the object that should be in position
$5$ in position $20$, then I pay $\$15$; if I put it in position
$30$, I only pay $\$20$ because nothing costs more than $K=\$20$.
(\emph{10 points})

\ene

\section{Programming Exercises}

\bee

\i Implement PCA, and Kernel PCA with RBF kernel. For the RBF kernel,
there is a fairly decent heuristic for choosing the bandwidth parameter: 
compute the pairwise distances (Euclidean but don't square it) between all 
the points and take the \emph{median} Euclidean distance as the bandwidth
parameter. 

The shell for both PCA and Kernel PCA are provided ({\tt PCA.m} and 
{\tt KPCA.m}) and you will need to fill in the TODO parts. 
Once you complete implementing these (they are independent), you can test them
on the MNIST digits dataset (provided in {\tt mnist.mat}).

The test scripts {\tt test\_pca.m} and {\tt test\_kpca.m} would test your
implementations and will produce several plots, including the figures of 
reconstructed digits using the projected data. Observe (for both
PCA and KPCA) how the reconstruct digits look like as you increase 
the number of dimensions $d$ the data is projected on.

Submit the completed codes for PCA and KPCA, and all the plots. Also comment on your
observations about the reconstruction accuracies of digits as $d$ varies.
(\emph{30 points})

\i In this task, we will implement AdaBoost.  A shell for AdaBoost is in
{\tt Boost.m}.  You will need to fill in the {\tt BoostTrain} and {\tt
  BoostPredict} functions.  See the comments in {\tt Boost.m} for a
description of what each function should do and how it should store
its outputs.  Note that in order to do boosting properly, you need to
have binary classification algorithms that can accept weighted data.
I have provided an implmentation of Perceptron that does so in
{\tt perceptron\_weighted.m} (do not change that code! :) ).

The data for this part is in {\tt data.mat}. When you do {\tt load data.m},
you would see {\tt trX,trY} for the training data and {\tt teX,teY} for
the test data.

Once you are done implementing {\tt Boost.m}, you can run {\tt test\_adaboost.m} 
which will run a series of tests using different number of rounds for the boosting 
algorithm. To get a sense of whether your implementation is doing sensibly, 
here are some representative numbers: for boosting round 1, you should get 
about 62.5\% training and 55.6\% test accuracy; for round 3, you should 
get training accuracy of about 88.5\% and 86.8\% test accuracy.
(\emph{20 points})

\ene

\end{document}
