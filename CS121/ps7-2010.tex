\documentclass[ps, letterpaper]{cs121}
\setlength{\parindent}{0in}
\begin{document}
\header{7}{Friday, November 12, 2009}{Monday, November 15, 2009}{20}

\newcommand{\<}{\langle}
\renewcommand{\>}{\rangle}

%%%%%%%%%%%%%%%%%%%%%%%%%%%%%%%%%%%%%%%%%%%%%%%%
% PROBLEM 1 - 
%%%%%%%%%%%%%%%%%%%%%%%%%%%%%%%%%%%%%%%%%%%%%%%%
\problem{10} %997A
Let $L_1 = \{\langle M \rangle : M  \mbox{ accepts } \langle M \rangle\}$ and
$L_2 = \{\langle M \rangle: M \mbox{ rejects } \langle M \rangle \}$. Prove that there is no recursive language $R$ such that $L_1 \subseteq R$ and $L_2 \subseteq \overline{R}$. (Hint: Suppose there were, and think about the TM that supposedly decides $\overline{R}$.)

%% \subsolution <your answer here>

%%%%%%%%%%%%%%%%%%%%%%%%%%%%%%%%%%%%%%%%%%%%%%%%
% PROBLEM 2 - 
%%%%%%%%%%%%%%%%%%%%%%%%%%%%%%%%%%%%%%%%%%%%%%%%
\problem{10}
Let $L_1$ be a language. Prove that $L_1$ is r.e.~if and only if there exists some recursive language $L_2$ such that $L_1 =\{x:$ there exists $y$ such that $\left<x,y\right> \in L_2 \}$. (Hint: Imagine $y$ gives you some information about an accepting computation on $x$, if one exists.)

%% \subsolution <your answer here>

%%%%%%%%%%%%%%%%%%%%%%%%%%%%%%%%%%%%%%%%%%%%%%%%
% PROBLEM 3 - 
%%%%%%%%%%%%%%%%%%%%%%%%%%%%%%%%%%%%%%%%%%%%%%%%
\problem{10} %%947
\vspace{-6pt}
A function $f:\Sigma^* \rightarrow \Sigma^*$ is {\it computable} if
there exists a Turing Machine $M$ that, when given $s \in \Sigma^*$ as input,
halts with $f(s)$ written on the tape. The \emph{range} of $f$ is $\{f(x):
  x \in \Sigma^*\}$. Prove that a nonempty language is r.e.~if and only if 
  it is the range of a computable function. 
 \vspace{-6pt}

%% \subsolution <your answer here>

%%%%%%%%%%%%%%%%%%%%%%%%%%%%%%%%%%%%%%%%%%%%%%%%
% PROBLEM 4 - 
%%%%%%%%%%%%%%%%%%%%%%%%%%%%%%%%%%%%%%%%%%%%%%%%
\problem{5+10+10}
In this problem you will define a function that grows faster than any computable function. Thus you will prove the existence of an uncomputable function directly, without relying on Turing's diagonalization argument. 
The {\bf busy-beaver function}
$\beta(n)$ is the
the largest number of $a$'s that can be printed by any $n$-state, two-symbol Turing machine that eventually halts when started from the empty tape.
\iffalse Formally, for each $n \in \Nat$, $\beta(n)$ is defined to be the largest $m$
such that, for some
one-tape Turing machine $M$ with alphabet $\{\sqcup, a\}$
and with exactly $n$
states,
$$(s,\triangleright \underline{\sqcup}) \vdash_M^*
  (h,\triangleright \underline{\sqcup} a^m).$$
\fi
  \subproblem
  Show that adding more states increases the number of $a$'s that can be
  written. Specifically, show that there is a constant $t$ such that, for all natural numbers $n$ and
  $m$,
  $$ \mbox{if } n \geq m +t \mbox{, then } \beta(n) > \beta(m).$$
  \vskip-4pt
  \subproblem
  Show that if $f:\Nat\rightarrow\Nat$ is computable, then there exists
  some constant $k_f$ such that
  $\beta(n+k_f)\geq f(n)$
  for all $n$.  In other words, there is an $(n+k_f)$-state Turing machine $M_n$
  that writes at least $f(n)$ $a$'s on a blank tape before
  halting.  ({\it Hint:}  First recall that any TM can be simulated by one with a two-symbol alphabet. Show that, for each $n$, there is an $(n+c)$-state TM
  $N_n$ that write $n$ $a$'s, where $c$ is
  a small positive constant independent of $n$.  Combine
  such a TM with the fixed two-tape-symbol machine $F$ which computes $f$.  The overall constant $k_f$ will represent
  the number of ``extra'' states $c$ required to construct $N_n$ plus
  the number of states of $F$.)
  \vskip4pt

  \subproblem
  % 
  % N.B. We removed the gamma(n) = beta(2n) hint this year
  % (see last year's problem set) and that was probably
  % a mistake.  People seemed to need that kick.
  %
  % What might be even more helpful is to explain in English the
  % ramifications of parts A and B so that students really get 
  % what's going on.  Something like
  %
  % A: Thus allowing only a constant number of additional states
  % is guaranteed to increase beta.
  %
  % B: Thus beta grows as fast or faster than _every_ computable
  % function.
  %
  Show that $\beta$ is not computable.  \iffalse For purposes of this problem, you may assume the
  following lemma: Any recursive function from numbers to numbers is
  computed by some Turing machine with alphabet $\{\sqcup,a \}$.\fi

%% \subsolution <your answer here>

%%%%%%%%%%%%%%%%%%%%%%%%%%%%%%%%%%%%%%%%%%%%%%%%
% PROBLEM 5 - 
%%%%%%%%%%%%%%%%%%%%%%%%%%%%%%%%%%%%%%%%%%%%%%%%
\problem{5 + 5} %%947
\vspace{-6pt}

In the near future you're working as an engineer at Google/Microsoft/Facebook when your manager asks you to write the following two programs. Is this a problem? Why or why not?

\subproblem Take another program's code as input and decide if that program is implemented in the fewest possible lines of code.

\subproblem Take another program's code and remove all inaccessible (dead) code from it.

%% \subsolution <your answer here>
%% \subsolution <your answer here>

%%%%%%%%%%%%%%%%%%%%%%%%%%%%%%%%%%%%%%%%%%%
% PROBLEM 6 - Challenge
%%%%%%%%%%%%%%%%%%%%%%%%%%%%%%%%%%%%%%%%%%%

\problem{\textbf{Challenge}  + 1}
Given a particular method of encoding a Turing machine $M$ into a string $\<M\>$, define $T_w$ to be the Turing machine encoded by the string $w$, or if $w$ is not a proper encoding of any Turing machine, then define $T_w$ to be an arbitrary fixed Turing machine. \iffalse That is, $$ T_w = \begin{cases} M & \text{ if $w = \<M\>$ for some TM $M$ } \\ 
         \text{The TM that always accepts} & \text{ otherwise} \end{cases} $$\fi
Prove that if $f$ is any computable function, then there exists some string $x$ such that $L(T_x) = L(T_{f(x)})$.

%% \subsolution <your answer here>

\end{document}
