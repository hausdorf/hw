\documentclass[letterpaper, ps]{cs121}
\usepackage{amsmath, amsfonts, graphicx}
\begin{document}
\header{2}{Friday, October 1, 2010}{Monday, October 4, 2010}{20}


%%%%%%%%%%%%%%%%%%%%%%%%%%%%%%%%%%%%%%%%%%%%%%%%%%%%%%%%%%%%%%%%%%%%%%%

% !!!!! Write your name here !!!!! %
\vspace{3mm}
\begin{center} \bf{Name} \end{center}
\studentName{!!! Your Name Here !!!}
\collaboratorNames{!!! Collaborators' names here !!!!}

%%%%%%%%%%%%%%%%%%%%%%%%%%%%%%%%%%%%%%%%%%%%%%%%%%%%%%%%%%%%%%%%%%%%%%%

\problem {2+2+2+2}
Translate the following languages over $\Sigma=\{a,b\}$ from English description to regular expressions, or vice versa.
\subproblem $L = \{w\in\Sigma^*:$ the second letter of $w$ is the same as the last letter, $|w| \geq 2 \}$
\subproblem $L = \{w\in\Sigma^*:$ every $a$ is followed by an odd number of $b$'s$\}$
\subproblem $(a(a\cup b)^*b) \cup (b(a\cup b)^*a)$
\subproblem $(b^*abbb^*) \cup (b^*bbab^*) \cup (b^*babb^*) \cup (bbb^*)$

%% \subsolution <Your answer goes here>
%% \subsolution <Your answer goes here>
%% \subsolution <Your answer goes here>
%% \subsolution <Your answer goes here>

%%%%%%%%%%%%%%%%%%%%%%%%%%%%%%%%%%%%%%%%%%%%%%%%%%%%%%%%%%%%%%%%%%%%%%%

\problem{4+4}
\subproblem Construct a DFA for $L = \{w \in \{0,1\}^* : \text{the number represented in binary notation by } w \text{ is equal to $1$ mod $3$}\}$.  That is, when we interpret $w$ as a number written in binary and divide by $3$, we get a remainder of $1$. Briefly explain why your DFA works.

Note: for this problem, we ignore leading zeros, so that $0111=7 \in L$.

\subproblem Convert your DFA for $L$ to a regular expression, using the GNFA construction described in lecture.  Show the full GNFA at each step of the construction.

%% \subsolution <Your answer goes here>
%% \subsolution <Your answer goes here>

\pagebreak

%%%%%%%%%%%%%%%%%%%%%%%%%%%%%%%%%%%%%%%%%%%%%%%%%%%%%%%%%%%%%%%%%%%%%%%

\problem{12} Let $\Sigma$ and $\Delta$ be alphabets. Consider a
function $\varphi: \Sigma \rightarrow \Delta^*$. Extend $\varphi$ to
a function from $\Sigma^* \rightarrow \Delta^*$ such that:
\begin{eqnarray*}
\varphi(\varepsilon) & = & \varepsilon \\
\varphi(w\sigma ) & = & \varphi(w)\varphi(\sigma ), \textrm{ for any }
w \in  \Sigma ^*, \sigma  \in  \Sigma
\end{eqnarray*}
For example, if $\Sigma = \Delta = \{a,b\},~\varphi(a) = ab,$ and
$\varphi(b) = aab,$ then
$\varphi(aab)=\varphi(aa)\varphi(b)=\varphi(a)\varphi(a)\varphi(b)=ababaab$.
 Notice how $\varphi$ operates on each input symbol individually. Any
function $\varphi:\Sigma^* \rightarrow \Delta^*$ defined in this way
from a function $\varphi: \Sigma \rightarrow \Delta^*$ is called a
{\bf
homomorphism}.\\\\
Prove that the set of regular languages is closed under
homomorphism.  Specifically, prove by induction on the length of $R$
that $\varphi[L(R)]$ is regular for any regular expression $R$ and
any homomorphism $\varphi$.  (In other words, prove that, if $L$ is
regular, then $\varphi[L]$ is also regular, for any homomorphism
$\varphi$.)

%%%%%%%%%%%%%%%%%%%%%%%%%%%%%%%%%%%%%%%%%%%%%%%%%%%%%%%%%%%%%%%%%%%%%%%

\problem{12}
Let $\mathprob{SUBSTRINGS}(L,A) = \{w \in L : \text{every substring of } w \text{ is in A}\}$. Prove that if $L$ and $A$ are regular, then so is $\mathprob{SUBSTRINGS}(L,A)$. Hint: can you make use of any closure properties we already know?

%%%%%%%%%%%%%%%%%%%%%%%%%%%%%%%%%%%%%%%%%%%%%%%%%%%%%%%%%%%%%%%%%%%%%%%

\problem{4+4+4+4}
\subproblem    Define the $\lesssim$ relation on nonempty sets by $S\lesssim T$ if 
there is an onto map from $T$ to $S$. \footnote{Students of logic may notice that the domain of this 
relation would be the set of all sets, which cannot be defined without 
leading to paradoxes.  The way around this is to talk about a relation on 
the `class' of all sets, but the distinction between the notions of a 
`class' and a `set' is beyond the scope of CS121.} Show that $S\lesssim T$ if and only if 
there is a one-to-one map from $S$ to $T$, so this relation is in fact the same as the one
that was defined in Problem Set 0.
\subproblem Show that for every set $S$, $S \lesssim P(S)$.
\subproblem   Show that for every nonempty set $S$, $P(S) \not\lesssim S$.
(Hint: The diagonalization technique does not require enumerating the set in question.) 
\subproblem Conclude that there are infinitely many different equivalence classes
under the relation $\sim$ defined in Problem Set 0, i.e. infinitely many different ``infinite cardinalities".

%% \subsolution <Your answer goes here>
%% \subsolution <Your answer goes here>
%% \subsolution <Your answer goes here>
%% \subsolution <Your answer goes here>


%%%%%%%%%%%%%%%%%%%%%%%%%%%%%%%%%%%%%%%%%%%%%%%%%%%%%%%%%%%%%%%%%%%%%%%

\problem{Challenge! 2+1}
\subproblem Let $L/A = \{ x : \text{$wx \in A$ for some $w \in L$} \}$. Show that if $A$ is regular and $L$ is \emph{any} language, then~$L/A$ is regular.
\subproblem Suppose $L = \{a^n : n \text { is greater than 2, is even, and cannot be expressed as the sum of two primes}\}$, and $A = \Sigma^*$. Why doesn't this contradict your proof?

%% \subsolution <Your answer goes here>
%% \subsolution <Your answer goes here>

%%%%%%%%%%%%%%%%%%%%%%%%%%%%%%%%%%%%%%%%%%%%%%%%%%%%%%%%%%%%%%%%%%%%%%%


\end{document}
