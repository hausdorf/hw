%Author Alex Clemmer
%CS 2100 Discrete Math
%Assignment 3:
\documentclass[a4paper]{article}
\usepackage[pdftex]{graphicx}
\usepackage{fancyvrb}
\usepackage{multirow}
\usepackage{amssymb}
\usepackage{amsmath}
\usepackage{fullpage}
\addtolength{\oddsidemargin}{-.05in}
	\addtolength{\evensidemargin}{-.05in}
	\addtolength{\textwidth}{.25in}

	\addtolength{\textheight}{.25in}

\begin{document}

\section*{Assignment 3}
Alex Clemmer\\
CS 2100 \\
Student number: u0458675

\subsection*{Problem 1:} 

So for $b_n = \cfrac{1}{2^1} + \cfrac{1}{2^2} + ... + \cfrac{1}{2^n}$. We want to show that $b_n = 1 - \cfrac{1}{2^n}$. Both formulae report that for $n=1$, $b = \frac{1}{2}$, and for $n = 2$, $b = \frac{3}{4}$. The recursive formula is as follows:

\begin{equation}
\begin{array}{rll}
b_n & = & b_{n-1} + \cfrac{1}{2^n} \\[.15in]
& = & \left( 1-\cfrac{1}{2^{n-1}} \right) + \cfrac{1}{2^n} \\[.2in]
1- \cfrac{1}{2^n} & = & \left( 1-\cfrac{1}{2^{n-1}} \right) + \cfrac{1}{2^n} \\[.15in]
1- \cfrac{2}{2^n} & = & 1-\cfrac{1}{2^{n-1}} \\[.15in]
\cfrac{2}{2^n} & = & \cfrac{1}{2^{n-1}} \\[.15in]
\end{array}
\end{equation}	

We can use this in the original equation:

\begin{equation}
\begin{array}{rll}
b_n & = & b_{n-1} + \cfrac{1}{2^n} \\[.15in]
& = & \left( 1- \cfrac{1}{2^{n-1}} \right) + \cfrac{1}{2^n} \\[.2in]
& = & 1- \cfrac{2}{2^n} + \cfrac{1}{2^n} \\[.2in]
& = & 1- \cfrac{1}{2^n} \\[.2in]
\end{array}
\end{equation}	

\subsection*{Problem 2:}

\begin{equation}
\begin{array}{rll}
(p_n - p_{n-1})^2 - 2p^2_{n-1} & = & (-1)^{n-1} \\[.05in]
p_n^2 - 2p_n p_{n-1} + p_{n-1}^2 - 2p_{n-1}^2 & = & \\[.05in]
p_n^2 - 2p_n p_{n-1} - p_{n-1}^2 & = & \\[.05in]
-p_n^2 + 2p_n p_{n-1} + p_{n-1}^2 & = & \\[.05in]
(p_n + p_{n-1})^2 - 2p_n^2 & = & (-1)^{n} \\[.05in]
2p_n^2 + (-1)^n & = & (p_n + p_{n-1})^2 \\[.05in]
2p_n^2 + (-1)^n & = & p_n^2 + p_n(p_{n-1}) + p_n^2 \\[.05in]
(-1)^n & = & -p_n^2 + p_n(p_{n-1}) + p_n^2 \\[.05in]
(-1)^{n+1} & = & (p_{n-1} - p_n)^2 \\[.05in]
(-1)^{n-1} & = & (p_{n-1} - p_n)^2 \\[.05in]
\end{array}
\end{equation}	

\subsection*{Problem 3:}

This next one is pretty simple.

\begin{equation}
\begin{array}{rll}
F_{2m-2} & = & F_1 + F_3 + ... + F_{2m-1} \\[.1in]
& = & F_1 + F_3 + ... + F_{2m-3} + F_{2m-1} \\[.1in]
& = & F_{2m-2} + F_{2m-1} \\[.1in]
\end{array}
\end{equation}	

This of course is the definition of the Fibonacci sequence.

\subsection*{Problem 4:}


\subsection*{Problem 5:}



\subsection*{Problem 6:}

If we divide up all numbers by $\equiv (\mbox{mod } 7)$, then we have 7 buckets, numbered 0 to 6. Given 5 numbers, either there must be 5 out of 7 buckets filled, or there must be one bucket for which $x_i > 1$. If its the latter case, then the rules holds, as any number that is subtracted by another number similarly $\equiv (\mbox{mod } 7)$ will give us $0 \equiv (\mbox{mod } 7)$ ($\textit{e.g.}$, $13-6 = 7$).

The $\textit{only}$ alternative is that at least one pair of the modulo sums up to seven by filling one of the critical pairs: $\{(1,6),(2,5),(3,4)\}$. In other words, failing the last condition, we $\textit{must}$ pick 5 numbers whose modulo are different. We cannot pick 5 modulo without satisfying at least one of the above pairs, and therefore this is true.

\subsection*{Problem 7:}

\paragraph{(c)} There are only 8 quadrants in a cube. The farthest points away from each other in any quadrant of a cube can be described as $a \sqrt{3}$. In this case, the side of the whole cube is length 1, which means that the length of the $\frac{1}{2}$. Thus the diagonals in any of the quadrants will be $\frac{\sqrt{3}}{2}$.

Since there are 9 points and only 8 quadrants, 2 points will be in the same quadrant by basic pigeonhole principle. Therefore, the farthest they can possibly be from each other is the length of the diagonal, $\frac{\sqrt{3}}{2}$.

\paragraph{(d)} There are four quadrants for any cartesian plane. With 5 points other than the origin $O$, there must be one quadrant with 2 points, by the basic pigeonhole principle. These points, called $P$ and $Q$, are guaranteed to make some acute angle $\angle POQ$, even if it's only slightly.

It has not escaped my attention that the boundaries may appear to present a problem, however, it should be noted that it is not correct to make the axes "communal" property among all quadrants -- in other words, we need to designate what quadrant each particular half-axis is a part of, otherwise their space is external to all the quadrants, which then makes "quadrants" an inaccurate name.

What's important is that the quadrants equally divide the plane, and that's what we assume here.

\end{document}