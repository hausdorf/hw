%Author Alex Clemmer
%CS 2100 Discrete Math
%Assignment 4:
\documentclass[a4paper]{article}
\usepackage[pdftex]{graphicx}
\usepackage{fancyvrb}
\usepackage{multirow}
\usepackage{amssymb}
\usepackage{amsmath}
\usepackage{fullpage}
\addtolength{\oddsidemargin}{-.05in}
	\addtolength{\evensidemargin}{-.05in}
	\addtolength{\textwidth}{.25in}

	\addtolength{\textheight}{.25in}

\begin{document}

\section*{Assignment 4}
Alex Clemmer\\
CS 2100 \\
Student number: u0458675

\subsection*{Problem 1:} 

\paragraph{(a)} I will assume that differences in order constitutes a "different combination", ($\textit{e.g.} \{1,1,2\} \not\equiv \{2,1,1\}$) because in this situation it makes sense, where in other situations (dice roll, for example), the answer is: $50 \cdot 50 \cdot 50 = 125,000$.

\paragraph{(b)} If we select one number and do not allow it to be selected again, then each subsequent number will have one less possibility. So: $50 \cdot 49 \cdot 48 = 117,600$.

\paragraph{(c)} The first can be any of the numbers, but the second must be the exact same number (which in other words means the second number can only ever be 1 number, which gives us $50 \cdot 1$ so far). I will also assume that the third must be different (which is a rational expectation given how we treated problems $\textit{a}$ and $\textit{b}$). Thus: $50 \cdot 1 \cdot 49 = 2,450$.

\paragraph{(d)} The strategy here is to add up all the possible ways that we could have two numbers that are the same and one that is not.

There are three ways this could happen: our pattern must be $XXY$, $XYX$, or $YXX$. The exclusive factor here is $Y$, because $Y$ must $\textit{not}$ be the same number as $X$. Thus for every possible $X$, there are 49 possible $Y$ (or, every number is possible $\textit{except}$ that one). Likewise, for every possible $X$, there is only one possible second $X$, since $X \ne X$ is not possible.

This makes the last part easy: for $XXY$, the combination must be $50 \cdot 1 \cdot 49 = 2,450$; for $XYX$, the combination must be $50 \cdot 49 \cdot 1 = 2,450$; for $YXX$, the combination must be $49 \cdot 50 \cdot 1 = 2,450$.

When we add these up, we get the total possibilities: $\textbf{7,350}$.

\begin{equation}
\begin{array}{rll}
b_n & = & b_{n-1} + \cfrac{1}{2^n} \\[.15in]
& = & \left( 1-\cfrac{1}{2^{n-1}} \right) + \cfrac{1}{2^n} \\[.2in]
1- \cfrac{1}{2^n} & = & \left( 1-\cfrac{1}{2^{n-1}} \right) + \cfrac{1}{2^n} \\[.15in]
1- \cfrac{2}{2^n} & = & 1-\cfrac{1}{2^{n-1}} \\[.15in]
\cfrac{2}{2^n} & = & \cfrac{1}{2^{n-1}} \\[.15in]
\end{array}
\end{equation}	

\end{document}