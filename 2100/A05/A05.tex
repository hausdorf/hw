%Author Alex Clemmer
%CS 2100 Discrete Math
%Assignment 5:
\documentclass[a4paper]{article}
\usepackage[pdftex]{graphicx}
\usepackage{fancyvrb}
\usepackage{multirow}
\usepackage{amssymb}
\usepackage{amsmath}
\usepackage{fullpage}
\addtolength{\oddsidemargin}{-.05in}
	\addtolength{\evensidemargin}{-.05in}
	\addtolength{\textwidth}{.25in}

	\addtolength{\textheight}{.25in}

\begin{document}

\section*{Assignment 5}
Alex Clemmer\\
CS 2100 \\
Student number: u0458675

\subsection*{Problem 1: \textit{13, pg. 423}} 

Where $S_1$, we can trivially see that the sequence is $0, 1$. Where $S_2$, we can see that the sequence is $00, 01, 10$. The thing to notice here is that there is a certain amount of repetitiveness --- that is, the sequence of $S_n = \{0, 1\}$ are the same as the first few derivations of $S_2$: the sequence is $\{00, 01, ...\}$, and if you ignore the trailing zeroes to the left, it is the same sequence.

The only question that would remain is, how fast does this sequence add new derivations to the sequence? The answer is that it adds $n-1$ numbers to the sequence $S_{n-1}$. Thus, our answer is:

\begin{equation}
\begin{array}{rll}
S_n = S_{n-1} + (n-1) \\[.0in]
\end{array}
\end{equation}

\subsection*{Problem 2: \textit{8, pg. 423}} 

$d_1 = 0$: if we have 1 envelope and 1 letter, then there are a total of 0 ways to put that letter in the wrong envelope. $d_2 = 1$: in the case that we have 2 envelopes and 2 letters, then there is exactly 1 way to put letters in incorrect envelopes. This in mind, it's relatively simple to derive $d_5$ from the formula:

\begin{equation}
\begin{array}{rcl}
d_5 & = & 4 \cdot (d_{4} + d_{3}) \\[.08in]
& = & 4 \cdot(d_4 + 2 \cdot (d_2 + d_1)) \\[.08in]
& = & 4 \cdot(d_4 + 2 \cdot (1 + 0)) \\[.08in]
& = & 4 \cdot(3 \cdot (d_3 + d_2) + 2) \\[.08in]
& = & 4 \cdot(3 \cdot (2 + 1) + 2) \\[.08in]
& = & 44 \\[.08in]
\end{array}
\end{equation}

\subsection*{Problem 3: \textit{10, pg. 423}}

Once again $t_n$ is a superset of $t_{n-1}$: for example, $t_1 = \{1-2\}$; observe that $t_2 = \{(1-2, 3-4), (1-3, 2-4), (1-4, 2-3)\}$ contains this pairing as the first element.

The next thing to observe is that the problem is asking us how many ways we can set up the first round. The easiest way to visualize this is to set up each of the potential matches into columns. I've organized $t_2$ as an example below. In this example, each player gets a number from 1 to 4, and each column represents a potential set of matches that might make up the first round.

\begin{verbatim}
for n = 2
round 1   round 2   round 3
1-2       1-3       1-4
3-4       2-4       2-3
\end{verbatim}

We can see here that there are thus a total of 3 ways to set up the first round of a tournament with 4 people in it. In contrast, notice how many matches are possible if there are only 2 players in the tournament:

\begin{verbatim}
round 1
1-2
\end{verbatim}

To solve this problem, the best way is to look at how $t_1$ is expanded to $t_2$, and how $t_2$ is expanded to $t_3$, and so on. I won't bother to write out $n=3$ because it's a bit larger, but it's just a single-dimensional transform.

That is, if we are recursively defining $t_n$, then the table should expand by a factor of $2n-1$. We can check this by observing that in expanding from 2 to 4 players, player 1 gets paired with all the others: from $\texttt{1-2}$ to $\texttt{1-2, 1-3, 1-4}$. When we multiply by $2n-1$, we are asserting this occurrence across all the numbers, which is accurate. Thus:

\begin{equation}
t_n = t_{n-1} \cdot (2n-1)
\end{equation}

\subsection*{Problem 4: \textit{6 parts (b) \& (d), pg. 435}} 

\paragraph{(b)} This problem is extremely straightforward. $\textit{Given}$:

\begin{equation*}
\begin{array}{rcl}
a_n & = & b \cdot a_{n+1} + c, n \ge 1 \mbox{ and } b \ne 1 \\[.08in]
\end{array}
\end{equation*}

We can derive the closed formula as follows:

\begin{equation*}
\begin{array}{rcl}
& = & L \cdot b^n + \cfrac{c}{1-b} \\[.08in]
\end{array}
\end{equation*}

Thus, given that $a_1 = 1, a_n = 3a_{n-1} + 2, n \ge 1$, we can use the above principle to derive most of the closed formula:

\begin{equation*}
\begin{array}{rcl}
a_n & = & L \cdot 3^n + \cfrac{2}{1-3} \\[.08in]
& = & L \cdot 3^n - 1 \\[.08in]
\end{array}
\end{equation*}

All that's left is to derive $L$, which is pretty trivial. We'll solve for $n = 2$.

\begin{equation*}
\begin{array}{rcl}
5 & = & L \cdot 3^1 - 1 \\[.0in]
6 & = & L \cdot 9 \\[.0in]
\cfrac{2}{3} & = & L \\[.08in]
\end{array}
\end{equation*}

And thus we have our final answer:

\begin{equation}
\begin{array}{rcl}
a_n & = & \cfrac{2}{3} \cdot 3^n - 1 \\[.08in]
\end{array}
\end{equation}

\paragraph{(d)} This problem follows directly from above. $\textit{Given } \mathit{a_1 = 3}$:

\begin{equation*}
\begin{array}{rcl}
a_n & = & 3 - a_{n-1} \\[.03in]
& = & L \cdot (-1)^n + \cfrac{3}{1+1} \\[.08in]
- \cfrac{3}{2} & = & L \cdot (-1)^n \\[.08in]
\end{array}
\end{equation*}

Given that $n=2$:

\begin{equation*}
\begin{array}{rcl}
- \cfrac{3}{2} & = & L \cdot (-1)^2 \\[.08in]
- \cfrac{3}{2} & = & L \\[.08in]
\end{array}
\end{equation*}

Thus:

\begin{equation}
\begin{array}{rcl}
a_n & = & - \cfrac{3}{2} \cdot (-1)^n + \cfrac{3}{2} \\[.08in]
\end{array}
\end{equation}

\subsection*{Problem 5: \textit{8 parts (a), (b) \& (c), pg. 436}} 

\paragraph{(a)} We can represent this situation in the following way:

\begin{equation*}
\begin{array}{rcl}
b_n & = & M_{n-1} + 0.10 \cdot M_{n-1}  \\[.08in]
\end{array}
\end{equation*}

Given what we have established above, it is possible to characterize this equation in the following closed format:

\begin{equation*}
\begin{array}{rcl}
b_n & = & L(1.1)^n \\[.1in]
\end{array}
\end{equation*}

Solving for L, we get:

\begin{equation*}
\begin{array}{rcl}
L & = & 909090.\overline{90} \\[.08in]
\end{array}
\end{equation*}

Which leaves our equation at:

\begin{equation*}
\begin{array}{rcl}
b_n & = & 909090.\overline{90} \cdot (1.1)^n \\[.08in]
\end{array}
\end{equation*}

Solving for $2,000,000$, we get $\mathbf{n=8.27254}$.

\paragraph{(b)} This equation will be a lot like the other one, the major difference being that we will subtract $50,000$ bacteria each time we update. Let's start with the recursive equation:

\begin{equation*}
\begin{array}{rcl}
b_n & = & M_{n-1} + 0.10 \cdot M_{n-1} - 50,000 \\[.08in]
\end{array}
\end{equation*}

This corresponds to the following iterative base equation:

\begin{equation*}
\begin{array}{rcl}
b_n & = & L \cdot 1.10^n - \cfrac{50,000}{1-1.1 } \\[.15in]
& = & L \cdot 1.10^n + 500,000 \\[.08in]
\end{array}
\end{equation*}

Use this to derive $L$:

\begin{equation*}
\begin{array}{rcl}
500,000 & = & L \cdot 1.10^n \\[.08in]
500,000 & = & L \cdot 1.10^n \\[.08in]
454545 & = & L \\[.08in]
\end{array}
\end{equation*}

Which gives us the full equation:

\begin{equation*}
\begin{array}{rcl}
b_n & = & 454545 \cdot (1.1)^n + 500,000 \\[.08in]
\end{array}
\end{equation*}

Solving for $2,000,000$, we get $\mathbf{n = 12.5267}$

\paragraph{(c)} This is more or less the same question as before.

\begin{equation*}
\begin{array}{rcl}
b_n & = & M_{n-1} + 0.10 \cdot M_{n-1} - 200,000 \\[.08in]
\end{array}
\end{equation*}

This leads to the iterative equation:

\begin{equation*}
\begin{array}{rcl}
b_n & = & L \cdot 1.10^n - \cfrac{200,000}{1-1.1 } \\[.15in]
& = & L \cdot 1.10^n + 2,000,000 \\[.08in]
\end{array}
\end{equation*}

Use this to derive $L$:

\begin{equation*}
\begin{array}{rcl}
-1,000,000 & = & L \cdot 1.10^n \\[.08in]
-1,00,000 & = & L \cdot 1.10^n \\[.08in]
-909090.\overline{90} & = & L \\[.08in]
\end{array}
\end{equation*}

Which gives us the full equation:

\begin{equation}
\begin{array}{rcl}
b_n & = & -909090.\overline{90} \cdot (1.1)^n + 2,000,000 \\[.08in]
\end{array}
\end{equation}

$\textbf{The impact is}$ that we deplete to 0 bacteria because the rate at which we take them is faster than the rate by which they populate the culture.

\subsection*{Problem 6: \textit{19 part (b), pg. 436}}

\paragraph{(b)} The given equation $a_n = 2a_{n-1} + 2a_{n-1}$ actually plugs directly into Theorem 4 from the book:

\begin{equation*}
\begin{array}{rcl}
x^2 & = & 2x + 2 \\[.08in]
\end{array}
\end{equation*}

The solutions to this quadratic are:

\begin{equation*}
\begin{array}{rcl}
x & = & \pm \sqrt{3} + 1 \\[.08in]
\end{array}
\end{equation*}

Plugging this back in, we get:

\begin{equation}
\begin{array}{rcl}
a_n & = & C \cdot (\sqrt{3} + 1)^n + K \cdot (-\sqrt{3} + 1)^n \\[.08in]
\end{array}
\end{equation}

\subsection*{Problem 7: \textit{22 part (c), pg. 436}}

We have an equation $a_n = 8a_{n-1} - 16a_{n-2}$ which corresponds to the quadratic $x^2 = 8x-16$. We can trivially find the root here, which is $x=4$. This gives us the final equation:

\begin{equation}
\begin{array}{rcl}
a_n & = & (C + K \cdot n) 4^n \\[.08in]
\end{array}
\end{equation}

\subsection*{Problem 8:}

So we already know that $A_i$ counts all distributions with at least $i+1$ balls in $i$-th box; this is of course useful for calculating the sets that do $\textit{not}$ fit the criteria delineated by the problem. Thus, we are able to derive (following notes 4.9 and 4.11 almost exactly):

\begin{equation}
\# \bigcup - (\#A_1 + \#A_2 + \#A_3) + (\#A_1 \cap \#A_2 + \#A_2 \cap \#A_3 + \#A_1 \cap \#A_3) \\
- (\#A_1 \cap \#A_2 \cap \#A_3)
\end{equation}

The important thing to note here is that, given the formula $\#A_i = \displaystyle \binom{5-(i+1)+3-1}{3-1}$, we know that the $\#A_1 \cap \#A_2 \cap \#A_3 = 0$, since if we tried to put $i+1$ balls in every single $i$-th container, we'd clearly run out of balls. This holds also for $\#A_1 \cap \#A_3$ and $\#A_2 \cap \#A_3$, but not for $\#A_1 \cap \#A_2$, which is $1$.

Because of this, we can derive our final answer:

\begin{equation}
\# \overline{\bigcup_{i=1}^3 A_i}  = \displaystyle \binom{5+3-1}{3-1} - \left( \binom{3+3-1}{3-1} + \binom{2+3-1}{3-1} + \binom{1+3-1}{3-1} \right) + (1+0+0) - 0 = 3
\end{equation}

And when you think about it, that's a lot of work to do to get an answer of $3$.

\subsection*{Problem 9:}

Now the trickier one. $A_1$ should be all the arrangements with ``00"; $A_2$ should be the arrangements with ``11"; $A_3$ should be all arrangements with ``22"; and $A_4$ should thus be all arrangements with ``33".

Given this, and given the notes online, we can say that $\#\bigcup = \cfrac{8!}{(2!)^4}$, $\#A_i = \cfrac{7!}{(2!)^3}$, $\#A_i \cap A_j = \cfrac{6!}{(2!)^2}$, $\#A_i \cap A_j \cap A_k = \cfrac{5!}{2!}$, and $\#A_1 \cap A_2 \cap A_3 \cap A_4 = 4!$.

Even though this question looks a lot more intimidating than the last one, the conclusion follows directly from these terms:

\begin{equation}
\# \overline{\bigcup_{i=1}^4 A_i}  = \displaystyle \binom{4}{0} \cfrac{8!}{(2!)^4} - \binom{4}{1} \cfrac{7!}{(2!)^3} + \binom{4}{2} \cfrac{6!}{(2!)^2} - \binom{4}{3} \cfrac{5!}{2!} + \binom{4}{4} 4! = 864
\end{equation}

\end{document}