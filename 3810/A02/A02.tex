%Author Alex Clemmer
%CS 3810 Computer Organization
%Assignment2:
\documentclass[a4paper]{article}
\usepackage[pdftex]{graphicx}
\usepackage{fancyvrb}
\usepackage{multirow}
\usepackage{amssymb}

\begin{document}

\section*{Assignment 2 }
Alex Clemmer\\
Student number: u0458675

\subsection*{Problem 1:}
\paragraph*{2.10.1:} Opcodes are the first 6 bits of an instruction. $Opcode_{(a)}$ is $\texttt{10 1011}_{two}$, which encodes $\texttt{sw}$. $Opcode_{(b)}$ is $\texttt{10 0011}_{two}$, which encodes $\texttt{lw}$.

\paragraph*{2.10.2:} Both $Instruction_{(a)}$ ($\texttt{sw}$) and $Instruction_{(b)}$ ($\texttt{lw}$) are I-type.

\paragraph*{2.10.3:} Hex conversions should be done from the least significant bit up towards the most significant bit: $Instruction_{(a)}$ becomes $\texttt{0xAD100002}$, and $Instruction_{(b)}$ becomes $\texttt{0xFFFFB353}$.

\paragraph*{2.10.4:} Let's first break each instruction into fields, starting with $Instruction_{(a)}$:

\begin{center}
\begin{tabular}{|c|c|c|c|c|c|}
\hline
opcode & rs & rt & rd & shamt & funct \\
\hline
\hline
add & \$t1 & \$t2 & \$zero & \multicolumn{2}{|c|}{} \\
\hline
00 0000 & 0 1000 & 0 0000 & 0 1000 & 0 0000 & 10 0000 \\
\hline
\end{tabular}
\end{center}

Split into bytes and converted into hex, we get $\texttt{0x01004020}$. The same follows for $Instruction_{(b)}$:

\begin{center}
\begin{tabular}{|c|c|c|c|}
\hline
opcode & rs & rt & constant or address \\
\hline
\hline
lw & \$t1 & \$s3 & 4 \\ 
\hline
10 0011 & 1 0011 & 0 1001 & 0000 0000 0000 0100 \\
\hline
\end{tabular}
\end{center}

In hex, that's $\texttt{0x8E690004}$.

\paragraph*{2.10.5:} $Instruction_{(a)}$ is R-type, while $Instruction_{(b)}$ is I-type.

\paragraph*{2.10.6:} According to the data we generated in the last few questions, $Instruction_{(a)}$ (which is R-type) has an opcode is $\texttt{0x00}$, an rs of $\texttt{0x08}$, an rt of $\texttt{0x00}$, an rd of $\texttt{0x08}$, a shamt of $\texttt{0x00}$ and a funct of $\texttt{0x20}$

$Instruction_{(b)}$ (I-type) has an opcode of $\texttt{0x23}$, an rs of $\texttt{0x13}$, an rt of $\texttt{0x09}$, and an immediate field of $\texttt{0x0004}$

\subsection*{Problem 2:}
\paragraph*{2.11.1:} Conversion from hex to binary is pretty simple. (a)'s $\texttt{0xAE0BFFFC}$ becomes $\texttt{1010 1110 0000 1011 1111 1111 1111 1100}_{two}$. (b)'s $\texttt{0x8D08FFC0}$ becomes $\texttt{1000 1101 1001 1000 1111 1111 1100 1001}_{two}$.

\paragraph*{2.11.2:}  (a) converted to a signed decimal is $\texttt{-1374945284}$ and (b) is (unsigned) $\texttt{2375614409}$.

\paragraph*{2.11.3:} Well, technically, (a) represents $\texttt{sw \$s0, 65532(\$t3)}$, but that immediate is huge and would get mitigated into multiple commands, such as the following: $\texttt{ori \$1, \$0, 0xfffc; addu \$1, \$1, \$11; sw \$16, 0x0000(\$1)}$. [Ed. note: I know there are no semicolons in MIPS; I use them here for conciseness, not because they actually exist.]

(b), is in a similar situation: $\texttt{lw \$t4, 65481(\$t8)}$ is the official translation, but this would end up getting mitigated for sure to something like $\texttt{ori \$1, \$0, 0xffc9; addu \$1, \$1, \$24; lw \$12, 0x0000(\$1)}$.

\subsection*{Problem 3:}
\paragraph*{2.17.1:} Complicated high-level procedures are characteristics of high-level languages. Pursuant to von Neumann's "simplicity" of "equipment", MIPS (and almost every other architecture, ever) doesn't include instructions like $\texttt{abs}$. Two or more smaller, simpler instructions make not only for faster execution, it leasts to much less complicated architecture design.

The reason $\texttt{sgt}$ doesn't exist is because it's redundant. In high-level code, it's more expressive and convenient, but in assembly it's neither necessary nor important. What's important is simple architecture and swift execution. So it's excluded.

\paragraph*{2.17.2:} Both $\texttt{abs}$ and $\texttt{sgt}$ would be R-type.

\paragraph*{2.17.2:} My implementation of $\texttt{sgt}$ might be something like this:

\begin{Verbatim}[fontsize=\small]
	slt $t1,$t2,$t3
	beq $t1, $zero, SetTrue
	li $t1, 0
	j AfterSetTrue
SetTrue:
	li $t1, 1
AfterSetTrue:
	[...]
\end{Verbatim}

On the other hand, $\texttt{abs}$ is a bit less bookkeep-y:

\begin{Verbatim}[fontsize=\small]
	slt $t2, $t3, $zero
	bne $t2, $zero, IsPositive
	nor $t2, $t3, $zero
	addi $t2, $t2, 1
IsPositive:
	[...]
\end{Verbatim}



\subsection*{Problem 4:}
\begin{Verbatim}[fontsize=\small]
# Author: Alex Clemmer
# Date:   9/4/10
#
# A MIPS program that will convert an integer into a hexadecimal number.
# Students must add the missing lines of code.


        .data

Prompt:
        .asciiz "Enter an integer: "

Output1:
        .asciiz "The decimal number "
        
Output2:
        .asciiz " is 0x"

Output3:
        .asciiz " in hexadecimal."

        
        .text

# Get input from user.  First, issue a prompt using the
# system call that will print out a string of characters.

        la $a0, Prompt  # Put the address of the string in $a0
        li $v0, 4
        syscall

# Next, make the system call that will wait for the user to input
# an integer.

        li $v0, 5  # Code for input integer
        syscall

# Integer input comes back in $v0
# Save the inputted integer in a saved register - important!
# It cannot stay in $v0 as we need to reuse $v0.

        move $s0, $v0

# Next, begin to output the result message.  This is done in several
# steps, including outputting strings and the original integer.

# Output first string.
        
        la $a0, Output1
        li $v0, 4
        syscall

# Output original integer
        
        move $a0, $s0  # Remember, $s0 contains the input number.
        li $v0, 1
        syscall

# Output second string.
        
        la $a0, Output2
        li $v0, 4
        syscall

# Output the hexadecimal number.  (This is done by isolating four bits
# at a time, adding them to the appropriate ASCII codes, and outputting
# the character.  It is important that the digits are output in
# most-to-least significant bit order.
        

                  # Set up a loop counter
	li $s1, 8
Loop:
                  # Roll the bits left by four bits - wraps highest bits to lowest bits (where we need them!)
	rol $s0, $s0, 4 # rol
                  # Mask off low bits (logical AND with 000...01111)
	andi $t0, $s0, 0xF # mask
                  # Determine if the bits represent a number less than 10 (slti)
	slti $t1, $t0, 10 # Is character less than?
                  # If not (if greater than 9), go make a high digit
	beq $t1, $zero, MakeHighDigit

MakeLowDigit:           
                  # Combine it with ASCII code for '0', becomes 0..9 
	addi $t0, $t0, 48 # add 0
                  # Go output the digit
	j DigitOut

MakeHighDigit:
                  # Subtract 10 from low bits
	subi $t0, $t0, 10
                  # Add them to the code for 'A' (65), becomes a..f
	addi $t0, $t0, 65

DigitOut:
        move $a0, $t0      # Output the ASCII character
        li $v0, 11
        syscall
        
                  # Decrement loop counter
	subi $s1, $s1, 1 # decrement loop count
                  # Keep looping if loop counter is not zero
	bne $s1, $zero, Loop

# Output third string.
        
        la $a0, Output3
        li $v0, 4
        syscall
        
        
# Done, exit the program

        li $v0, 10  # This system call will terminate the program gracefully.
        syscall
\end{Verbatim}

\end{document}