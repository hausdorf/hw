\documentclass[11pt]{article}
\usepackage{euscript}

\usepackage{amsmath}
\usepackage{amsthm}
\usepackage{amssymb}
\usepackage{epsfig}
\usepackage{xspace}
\usepackage{color}
\usepackage{url}

%%%%%%%%%%%%%%%%%%%%%%%%%%%%%%%%%
\setlength{\textheight}{9in}
\setlength{\topmargin}{-0.600in}
\setlength{\headheight}{0.2in}
\setlength{\headsep}{0.250in}
\setlength{\footskip}{0.5in}
\flushbottom
\setlength{\textwidth}{6.5in}
\setlength{\oddsidemargin}{0in}
\setlength{\evensidemargin}{0in}
\setlength{\columnsep}{2pc}
\setlength{\parindent}{1em}
%%%%%%%%%%%%%%%%%%%%%%%%%%%%%%%%%


\newcommand{\eps}{\varepsilon}

\renewcommand{\c}[1]{\ensuremath{\EuScript{#1}}}
\renewcommand{\b}[1]{\ensuremath{\mathbb{#1}}}
\newcommand{\s}[1]{\textsf{#1}}

\newcommand{\E}{\textbf{\textsf{E}}}
\renewcommand{\Pr}{\textbf{\textsf{Pr}}}

\title{HW07: More Bayes' Net
\footnote{\s{CS 5300 AI; \;\; Spring 2012 \hfill
Instructor: Jur van den Berg, University of Utah}
}
}
\author{Alex Clemmer, u0458675}

\begin{document}
\maketitle





%%%%%%%%%%%%%%%%%%%%%%%%%%%%%%%%%%%%%%%%%%%%%%%%%%%%
%%%%%%%%%%%%%%%%%%%%%%%%%%%%%%%%%%%%%%%%%%%%%%%%%%%%
%%%%%%%%%%%%%%%%%%%%%%%%%%%%%%%%%%%%%%%%%%%%%%%%%%%%
\section{D-Separation}

\paragraph{(a)} AV and SL are \textbf{not independent}.

\paragraph{(b)} AA and FP are \textbf{independent}.

\paragraph{(c)} U and BB are \textbf{independent}.

\paragraph{(d)} SL and BB are \textbf{not independent}.


\paragraph{1.} Assuming no evidence is given, nothing changes. \textbf{Not independent}.

\paragraph{2.} SL is a sink that connects AA and FP. \textbf{Not independent}.

\paragraph{3.} AA would actually ``block" the dependence. \textbf{Independent}.

\paragraph{4.} FP actually ``blocks" the dependence. \textbf{Independent}.




\section{Variable Elimination}

The joint distribution is represented by $P(a,b,c,d,e,f) = P(A)P(B|A)P(C|A)P(D|B)P(E|B)P(F|C)$. One way to think of our factored representation is as

\begin{equation}
P(a,b,c,d,e,f) = P(A) \sum_{b \in B} P(B|A) \sum_{c \in C} P(C|A) \sum_{d \in D, e \in E} P(D|B)P(E|B) \sum_{f \in F} P(F|C)
\end{equation}

Note that because each of these sums adds to 1, we can chop off an arbitrary number of factors to eliminate them. Thus our variable elimination results in $P(C=c | F=f, D= \sim d) = 0.6943$.



\section{Sampling}

\paragraph{1.} Of the $N = 20$ samples generated, there are 10 for which is is true that $S = s$. Of these there are 3 for which it is true that $A = a$, and 7 for which it is true that $A = \sim a$.

Having generated and rejected these samples, we must now normalize these counts. Normalizing them gives us $P(A=a|S=s) = \frac{3}{7+3} = 0.3$.

\paragraph{2.} The number of rejected samples, straightforwardly, is the number of samples for which it is \textit{not} true that $S = s$. In other words, it is quite literally: $P(S= \sim s)$.




\end{document}
