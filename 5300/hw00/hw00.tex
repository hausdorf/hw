\documentclass[12pt]{article}

\usepackage{times}
\usepackage{notes}
\usepackage{url}

\begin{document}
\lecture{Artificial Intelligence}{HW00: Student Survey}{Alex Clemmer u0458675, Spring 2011}

\begin{enumerate}

\item Which of the following courses have you taken:

      \begin{enumerate}
      \item Multivariate calculus
      \item Linear algebra
      \item Probability and statistics \textbf{Yes}
      \item Algorithms \textbf{Yes}
      \item Computer vision
      \item Natural language processing \textbf{Yes}
      \item Robotics
      \item Optimization (linear, quadratic, convex, etc.)
      \end{enumerate}

% ANY LINE BEGINNING "%" IS A COMMENT.  YOU CAN UNCOMMENT THE BELOW
% TEXT AND FILL IN YOUR OWN.
% \begin{solution}
%   John has taken all of these courses (and many more math courses)
% \end{solution}

\item List a few CS topics that interest you.

\begin{solution}
Massive-scale learning tasks, particularly those involving NLP. Randomized algorithms. Probabilistic graphical models. Statistical decision theory. Bayesian nonparametric analysis. Kernel methods. Convex optimization. Computational geometry.
\end{solution}

\item How would you rate your programming skills (1-10, 10 best)?  How
would you rate your math skills?  Are you familiar with scripting
languages like Python?

\begin{solution}
\textbf{Programming:} 7? I'm not terrible, but I'm not the best at the U either. \newline
\textbf{Math:} 7? I'm not the worst mathemetician in CS, but I'm \textit{definitely} not the best. \newline
\textbf{Scripting languages:} I'm proficient with Python and Ruby. I'm ok with Racket/Scheme and R.
\end{solution}

\item What are your goals in this class?

\begin{solution}
Mainly I want to approach a strong mathematical formalization of the methods and techniques commonly used in AI. Implementation experience would also be useful, but it's more important to me that I understand broadly the tenents of the field than it is that I implement something I marginally understand. For example, it would be horrible if I implemented some sort of PGM without understanding in detail how it works.
\end{solution}

\item Please be sure that you have subscribed to the
\verb+cs5300@list.eng.utah.edu+ mailing list.

\end{enumerate}

\end{document}
