%Author Alex Clemmer
%CS 4150 Algorithms
%Assignment 14:
\documentclass[a4paper]{article}
\usepackage[pdftex]{graphicx}
\usepackage{fancyvrb}
\usepackage{multirow}
\usepackage{amssymb}
\usepackage{amsmath}
\usepackage{fullpage}
\addtolength{\oddsidemargin}{-.05in}
	\addtolength{\evensidemargin}{-.05in}
	\addtolength{\textwidth}{.25in}

	\addtolength{\textheight}{.25in}
\begin{document}

\section*{Assignment 14}
Alex Clemmer\\
Student number: u0458675

\section{}
\subsection{}

The first implementation of $\texttt{explore()}$ calls $\texttt{postvisit()}$ before the children nodes are visited. This has the obvious effect of breaking the exploration of the dfs: note that in the book $\texttt{explore()}$ is called recursively on the child nodes before $\texttt{postvisit()}$ is called upon the final traversal of that particular node.

\section{}
\subsection{}

No it does not. It searches depth-first via the rightmost node. The book clearly goes from the leftmost node to the rightmost.

\subsection{}

In the second implementation of $\texttt{explore()}$, we only call $\texttt{postvisit()}$ in the event that we've pushed $\texttt{-v}$ to the stack for some $\texttt{v}$ we just visited. Before we actually pop that element, though, we will have pushed all that node's children to the stack, and all $\textit{that}$ node's children, and so on, until the first leaf. Only then can we recurse back and call that method, as only then will can we encounter a $\texttt{-v}$ on the stack.

\subsection{}

The worst case is $\Theta(|V|^2)$. If all the nodes are connected, then we will add every node for every node in the list. There is no alternative worst case, so this gives us not just $O(|V|^2)$, but also $\Theta(|V|^2)$.

Just like any $O(n)$ vs. $O(n^2)$ relationship, the $O(n^2)$ could represent an improvement over the linear algorithm if the linear algorithms constants are sufficiently high. In our case, for example, if the stack frames caused a prohibitive overhead, it may be some time before the $O(n^2)$ algorithm fell behind.

\subsection{}

Asymptotically, no. In a fully-connected graph, if we don't add elements that are visited, then each iteration will give us only one less than the previous total. This is a well-known relation: $\sum_{j} (n-j) = \frac{n(n-1)}{4}$, which is very clearly $O(n^2)$ So the end result is pretty much the same.




















\end{document}