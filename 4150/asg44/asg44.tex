%Author Alex Clemmer
%CS 4150 Algorithms
%Assignment 44:
\documentclass[a4paper]{article}
\usepackage[pdftex]{graphicx}
\usepackage{fancyvrb}
\usepackage{multirow}
\usepackage{amssymb}
\usepackage{amsmath}
\usepackage{fullpage}
\addtolength{\oddsidemargin}{-.05in}
	\addtolength{\evensidemargin}{-.05in}
	\addtolength{\textwidth}{.25in}

	\addtolength{\textheight}{.25in}
\begin{document}


\section*{Assignment 44}
Alex Clemmer\\
Student number: u0458675

\section{}

A problem is said to be $\in NP$ if we can solve it in nondeterministic polynomial time, or in other words, if there exists some algorithm $\in P$ that checks the solution of said problem. So all we really need to do is to show that there is some algorithm $\in P$ that checks the answer for this problem.

Fortunately, checking is stupid-simple: look through the $k$-tree we've generated, and check that no node has a degree $> k$. This is pretty clearly a polynomial-time deterministic algorithm.

\section{}

The challenge of reduction is to show that every problem in some class $T$ is expressible as a problem in some other class $Q$.

Fortunately, this is not a complicated reduction: if we have a Rudrata path, then each vertex must have 2 or fewer degrees, because each node is touched exactly once; if it had 3, at some point another path must pass through that vertex. Further, if we have a Rudrata path, it is a spanning tree, because we're touching every vertex $\textit{exactly}$ once.

So really it's the same problem. Since Rudrata is $NP$-complete, so is $k$-SpanningTree.

\section{}

This is the first part of the induction step in our inductive proof that deciding $k$-SpanningTree is $NP$ complete for $k \ge 2$. Basically our challenge is to show that any $k$ is reducible to $k+1$ for $k \ge 2$.

The classical proof? Given some graph $G$ with some 2-SpanningTree, add $k-2$ leaf vertices to every vertex, call this graph $G'$. If $G$ has a 2-SpanningTree, then $G'$ has a $k$-SpanningTree, since each node had 2 or fewer connected vertices, and we added $k-2$ to each of them. This shows that if there is no solution to the $k$ problem, there is also no solution to the corresponding $k-1$ problem.

In the other direction, given the $k$-SpanningTree $G'$, we can pull the $k-2$ leaf nodes from each and get a 2-SpanningTree back. This shows that if $k$ has a solution, we can determine it using a corresponding $k+1$ SpanningTree.

This shows that there is a $k$ to $k+1$ reduction: that is, for every $k$ problem, it can be mapped to a $k+1$ problem. This is the definition of a mapping reduction.

\section{}

Trivially follows from the above. We've shown that $k=2$ is $NP$-complete, and that every $k$ problem can be fully mapped and solved in terms of some $k+1$. Thus $\forall k \ge 2$, $k$-SpanningTree is $NP$-complete.























\end{document}