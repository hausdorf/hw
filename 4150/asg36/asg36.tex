%Author Alex Clemmer
%CS 4150 Algorithms
%Assignment 36:
\documentclass[a4paper]{article}
\usepackage[pdftex]{graphicx}
\usepackage{fancyvrb}
\usepackage{multirow}
\usepackage{amssymb}
\usepackage{amsmath}
\usepackage{fullpage}
\addtolength{\oddsidemargin}{-.05in}
	\addtolength{\evensidemargin}{-.05in}
	\addtolength{\textwidth}{.25in}

	\addtolength{\textheight}{.25in}
\begin{document}


\section*{Assignment 36}
Alex Clemmer\\
Student number: u0458675

\section{}

Greedily picking the odd number closest an even number will do just fine.

On a very small scale, this is easy to see: say $\texttt{\{1,2\}}$, the distance will always just be 1.

The next largest sequence would be digits [1,4], $\textit{e.g.}$, $\texttt{\{1,3,2,4\}}$. This particular sequence is actually the worst case, with a total distance of 4. We could also have something like $\texttt{\{1,2,3,4\}}$, with a total distance of 2.

The reason this works mainly because there are an equal number of even and odd numbers. So for some set of numbers, for every number that has a distance of $n$ from its closest correspondent, there will be another with the same distance $n$. This means that picking any choice balances out the corresponding choice if it's not optimal.

















\end{document}