%Author Alex Clemmer
%CS 4150 Algorithms
%Assignment 11:
\documentclass[a4paper]{article}
\usepackage[pdftex]{graphicx}
\usepackage{fancyvrb}
\usepackage{multirow}
\usepackage{amssymb}
\usepackage{amsmath}
\usepackage{fullpage}
\addtolength{\oddsidemargin}{-.05in}
	\addtolength{\evensidemargin}{-.05in}
	\addtolength{\textwidth}{.25in}

	\addtolength{\textheight}{.25in}
\begin{document}

\section*{Assignment 11}
Alex Clemmer\\
Student number: u0458675

\section*{}

The assertion that $\mbox{log}(n!) \in \Omega(n \mbox{ log } n)$ is true essentially because the square of half of $n$ will always be less than $n!$. The assertion that is provided statement taking the ``hint" in a very literal sense.

To prove this, we are essentially showing that $\exists c : \mbox{log }(n!) \ge c \cdot (n \mbox{ log } n)$ where $n >$ some $n_0$. This is more intimidating than it sounds. Another way of writing it is that $n \ge p^p, \forall n : n \ge n_0, n = 2p$. To illustrate the given claim, we start with some $n_0 = 6$:

\begin{equation*}
\begin{array}{rcl}
n_0! & \ge & p^p \\
6 \cdot 5 \cdot 4 \cdot 3 \cdot 2 \cdot 1 & \ge & 3^3 \\
720 & \ge & 27 \\
\end{array}
\end{equation*}

This is the basis. The intuition is that this will always be the case, for any $n$. But is it true?

Yes it is. First, consider that $n!$ is always a series of $n$ terms, while $p^p$ is a series of $n/2$ terms ($\textit{e.g.}$ for $n = 6, n! = 6 \cdot 5 \cdot 4 \cdot 3 \cdot 2 \cdot 1$, while $p^p = 3 \cdot 3 \cdot 3$). More importantly, however, is that $n/2$ of these terms are $\textit{always}$ bigger than all the terms in $p^p$.  So in the example above, $p^p = 3 \cdot 3 \cdot 3$, and the first 3 terms of $n!$ are $6 \cdot 5 \cdot 4$. So, even if we just had the first $n/2$ terms of $n!$, it would $\textit{still}$ be larger than $p^p, \forall p$.

Just like before, transforming this logic into the form of the assertion is trivial. Both are monotonically increasing, so the inequality will remain intact by log-ing both sides:

\begin{equation*}
\begin{array}{rcll}
n! & \ge & p^p & , \forall n > n_0, n = 2p \\
\mbox{ log }(n!) & \ge & p \mbox{ log } p & , \forall n > n_0, n = 2p \\
\end{array}
\end{equation*}

This is the definition of $\mbox{ log }(n!) \in \Omega(n \mbox{ log } n)$. Therefore $\mbox{ log }(n!) \in \Theta(n \mbox{ log } n)$.


















\end{document}