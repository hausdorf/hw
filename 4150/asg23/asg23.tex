%Author Alex Clemmer
%CS 4150 Algorithms
%Assignment 23:
\documentclass[a4paper]{article}
\usepackage[pdftex]{graphicx}
\usepackage{fancyvrb}
\usepackage{multirow}
\usepackage{amssymb}
\usepackage{amsmath}
\usepackage{fullpage}
\addtolength{\oddsidemargin}{-.05in}
	\addtolength{\evensidemargin}{-.05in}
	\addtolength{\textwidth}{.25in}

	\addtolength{\textheight}{.25in}
\begin{document}


\section*{Assignment 23}
Alex Clemmer\\
Student number: u0458675

\section*{}

We exclude negative cycles, because they are not well-defined for the shortest path problem.

That in mind, if all negative edges originate at origin $s$, then Dijkstra's should actually work. Consider what we would have to do if we happened to encounter a negative edge in some graph: to correct this, we would have to backtrack to make sure we weren't trampling over the $\textit{real}$ shortest path. This is because the path cost of a node that's already visited will change.

Note that if we can encounter this negative edge only at the very beginning, we never ever have to backtrack: the negative cost will change the cost of nodes that do not exist. Thus Dijkstra's should work if all negative edges originate from $s$.




















\end{document}