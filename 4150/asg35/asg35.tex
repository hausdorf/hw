%Author Alex Clemmer
%CS 4150 Algorithms
%Assignment 35:
\documentclass[a4paper]{article}
\usepackage[pdftex]{graphicx}
\usepackage{fancyvrb}
\usepackage{multirow}
\usepackage{amssymb}
\usepackage{amsmath}
\usepackage{fullpage}
\addtolength{\oddsidemargin}{-.05in}
	\addtolength{\evensidemargin}{-.05in}
	\addtolength{\textwidth}{.25in}

	\addtolength{\textheight}{.25in}
\begin{document}


\section*{Assignment 35}
Alex Clemmer\\
Student number: u0458675

\section{}

Right out of the box, suppose we have an undirected, weighted graph with 3 vertices:

\begin{verbatim}
a   b
o---o
 \  |
   o
   c
weight(a, b) = 1
weight(b, c) = 2
weight(a, c) = 3
\end{verbatim}

So if we partition the graph such that $G_1$ has the set of vertices $\{b\}$ and $G_2$ has the set of vertices $\{a, c\}$, then we would have to accept $(a, c) \in E$, which is an edge with a weight of 3. There are no self-loops in undirected graphs, so there is no edge $(b, b) \in E$. This leaves us with the bridging edge from $G_1$ to $G_2$. In this case it would be $(a,b) \in E$, which has a weight of 1. This gives us the following MST:

\begin{verbatim}
a   b
o---o
 \  
  o
   c
weight(a, b) = 1
weight(a, c) = 3
\end{verbatim}

But the actual MST should be:

\begin{verbatim}
a   b
o---o
    |
    o
   c
weight(a, b) = 1
weight(b, c) = 2
\end{verbatim}

\section{}

The cut property relies on the use of all vertices. If you establish the MST for two components they do not necessarily correspond to the MST of the entire tree.

\section{}

The recurrence relation is $T(n) = 2T(n/4) + O(n)$. This gives us $O(n)$. Even though the inner loop runs at ~$O(7n)$, what really matters is that each subproblem is 1/4th the size of its superproblem, that the work is done twice, and that the zip-up time is $O(n)$.









\end{document}