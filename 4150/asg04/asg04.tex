%Author Alex Clemmer
%CS 4150 Algorithms
%Assignment 04:
\documentclass[a4paper]{article}
\usepackage[pdftex]{graphicx}
\usepackage{fancyvrb}
\usepackage{multirow}
\usepackage{amssymb}
\usepackage{amsmath}
\usepackage{fullpage}
\addtolength{\oddsidemargin}{-.05in}
	\addtolength{\evensidemargin}{-.05in}
	\addtolength{\textwidth}{.25in}

	\addtolength{\textheight}{.25in}
\begin{document}

\section*{Assignment 04}
Alex Clemmer\\
Student number: u0458675

\section*{Hypothesis}

The assignments give us that adding elements into a static array is $O(k)$. This ``suggests" that the runtime for appending to a dynamic array would be ``the cost of doing 3k store operations" on some static array. We want to create and experiment to analyze this claim.

\section*{Procedure}

\subsubsection*{Foreward} One of the main weaknesses of my last assignment is that I did not have a systematic approach for creating the timing suite. This time, I did.

I started by just timing how long it took to allocate things to the array. Then I added code that spun the method for a few seconds before. Then I added code to time the loop itself, and subtract it from the total running time. At each point, I stopped and ran the program. If my code brought me closer to what I expected, I continued. If not, I fixed it.

The big insight here is that in my last assignment, I really just threw code onto the canvas. By incrementally developing the timing patterns, I was able to see in real time what did and did not work. This made my timing development both smoother and (probably) more correct.

\subsubsection*{How it works}

First, there were a couple of things that I found had no discernible effect on timing. First, it did not seem to matter whether I allocated one particular integer, or a lot of random integers. It also did not seem to matter, surprisingly enough, whether the integer was a variable or a literal.

One thing that did matter, a lot, was whether I used $\texttt{int}$ or $\texttt{Integer}$. I used the latter, but this may not have been the best choice. Either way, when I switched my static array from primitives to boxed primitives, the gap between their running time diminished from 9 times to ~2-3 times.

\end{document}