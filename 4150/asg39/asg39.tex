%Author Alex Clemmer
%CS 4150 Algorithms
%Assignment 39:
\documentclass[a4paper]{article}
\usepackage[pdftex]{graphicx}
\usepackage{fancyvrb}
\usepackage{multirow}
\usepackage{amssymb}
\usepackage{amsmath}
\usepackage{fullpage}
\addtolength{\oddsidemargin}{-.05in}
	\addtolength{\evensidemargin}{-.05in}
	\addtolength{\textwidth}{.25in}

	\addtolength{\textheight}{.25in}
\begin{document}


\section*{Assignment 39}
Alex Clemmer\\
Student number: u0458675

\section{}



One key insight here is that simply recursively finding all expressions that evaluate to $b$ will actually leave some solutions out over arbitrary problem-space, because other expressions, ($\textit{e.g.}$, $a \cdot a$) also evaluate to $b$.

So the real task at hand is to find a substructure to build the recurrence relation on. If we notice that multiplication in general is really a problem of tree associations, then this becomes easier. Say $M_x(i,j)$ yields a list of possible products of $i$ and $j$. Then $M_x(i,i) = s[i]$, and $M_x(i,i+1)$ is always deterministically one particular character from the given set of possibilities (in this case, $\{a, b, c\}$). And every subsequent problem is an expression of this. So $\textit{e.g.}, M_x(i, i+3)$ would be some combination of the above.

















\end{document}