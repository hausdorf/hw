%Author Alex Clemmer
%CS 4150 Algorithms
%Assignment 19:
\documentclass[a4paper]{article}
\usepackage[pdftex]{graphicx}
\usepackage{fancyvrb}
\usepackage{multirow}
\usepackage{amssymb}
\usepackage{amsmath}
\usepackage{fullpage}
\addtolength{\oddsidemargin}{-.05in}
	\addtolength{\evensidemargin}{-.05in}
	\addtolength{\textwidth}{.25in}

	\addtolength{\textheight}{.25in}
\begin{document}


\section*{Assignment 19}
Alex Clemmer\\
Student number: u0458675

\section{}

We have some set of currencies $C$ and some set of edges between them $E$---note that edges are NOT the raw exchange rates. To compute some edge $\in E$, we simply subtract the exchange rate corresponding to that edge by the complement. For example: the weight of $e_{ij} \in  = r_i - r_j$, and the weight of $w_{ji} = r_j - r_i$. In this way, one will be negative and the other will be positive. This lends itself well to Bellman-Ford.

The trick here is that each edge has some cost, and some are positive, while others are negative. This means that the ``best" path ends up being the one with the smallest net distance, or the ``shortest" path, which is classic Bellman-Ford terrain.

\section{}

Knowing that there is a positive cycle means that we can just exchange in sequence through that loop, adding some small delta of money each time until the loop disappears.

\section{}

Adapt the $G^R$ algorithm we learned. First, reverse the polarity and apply a depth-first search, saving the pre- and post-numbers as you go. Reverse the polarity again, and depth-first search again based on the post numbers, noting the weights as you go; if the sink has a net-path that's positive, then you have found a strongly-connected component that has a positive net weight. Simple.






















\end{document}