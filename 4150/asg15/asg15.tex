%Author Alex Clemmer
%CS 4150 Algorithms
%Assignment 15:
\documentclass[a4paper]{article}
\usepackage[pdftex]{graphicx}
\usepackage{fancyvrb}
\usepackage{multirow}
\usepackage{amssymb}
\usepackage{amsmath}
\usepackage{fullpage}
\addtolength{\oddsidemargin}{-.05in}
	\addtolength{\evensidemargin}{-.05in}
	\addtolength{\textwidth}{.25in}

	\addtolength{\textheight}{.25in}
\begin{document}

\section*{Assignment 15}
Alex Clemmer\\
Student number: u0458675

\section{}

If there are $2^{36}$ pages on the internet and there are $2^4$ links per page, then this is a dense graph, with $2^{40}$ edges.

An $\textbf{adjacency matrix}$ for a graph with $2^{36}$ nodes would be $2^{36} \times 2^{36} = 2^{72}$---that is, one element for every possible edge (which in this case would be a URL). This matrix will never be larger or smaller, because it is a 2-dimensional matrix representing all possible edges, whether or not there is an edge there.

Each element is just 1 bit, so $2^{72}$ bits is the total size for the elements in the table. Since arrays in Java and C are implemented much the same way a jump table is, beyond that, you really just need a pointer or reference to the first element, at which point you can jump to any space by telling the computer exactly how many elements to jump.

If a terabyte is $2^{43}$ bits, then we need $2^{29}$ terabytes to hold this array.

An $\textbf{adjacency list}$, on the other hand, would require $2^{36}$ base nodes, each with an average of 16 64-bit elements, which means that the total required space for just the base nodes is $2^{36} \cdot 2^6 \cdot 2^4 \cdot 2^{6} = 2^{52}$ bits = 512 terabytes.






















\end{document}