%Author Alex Clemmer
%CS 4150 Algorithms
%Assignment 25:
\documentclass[a4paper]{article}
\usepackage[pdftex]{graphicx}
\usepackage{fancyvrb}
\usepackage{multirow}
\usepackage{amssymb}
\usepackage{amsmath}
\usepackage{fullpage}
\addtolength{\oddsidemargin}{-.05in}
	\addtolength{\evensidemargin}{-.05in}
	\addtolength{\textwidth}{.25in}

	\addtolength{\textheight}{.25in}
\begin{document}


\section*{Assignment 25}
Alex Clemmer\\
Student number: u0458675

\section{}

They are relatively prime numbers, so we can just do EE:

\begin{verbatim}
(1, 0, 1)
(0, -5, 1)
(-5, 16, 1)
(16, -21, 1)
(-21, 16, 1)
\end{verbatim}

This of course gives us 46 as the modular mutiplicative inverse.

\section{}

There should be as many inverses as there are relative primes. We know this because EE requires that numbers be relatively prime.

















\end{document}