%Author Alex Clemmer
%CS 4150 Algorithms
%Assignment 12:
\documentclass[a4paper]{article}
\usepackage[pdftex]{graphicx}
\usepackage{fancyvrb}
\usepackage{multirow}
\usepackage{amssymb}
\usepackage{amsmath}
\usepackage{fullpage}
\addtolength{\oddsidemargin}{-.05in}
	\addtolength{\evensidemargin}{-.05in}
	\addtolength{\textwidth}{.25in}

	\addtolength{\textheight}{.25in}
\begin{document}

\section*{Assignment 12}
Alex Clemmer\\
Student number: u0458675

\section{}

We can describe this specific case of merge sort with the recurrence relation $T(n) = 2T(n/2) + O(1)$. Since $2 > 2^0$, we know that $O(n^{log_2 2}) = O(n^1)$. Note that I used a generalization of the Master Theorem here.

\section{}

We have $n$ sequential elements for some arbitrarily large $n$. Since we are representing them in bits, the comparisons will take $O(n \mbox{ log } n)$ comparisons, because representing a number takes log$_2$ bits, and you must compare each bit in the worst case for every $n$.

Unfortunately, this gives us an equation that doesn't really fit with the master theorem. What we do know is that $d > 1$. Since log$_2 2 = 1$, and $d > 1$, we know that $O(n^{d})$ for some $d > 1$.

\section{}

What we want to do is find a quadratic (much like Gauss's example) whose terms cancel out and leave us with only the product we seek. This turns out to be extremely simple:

\begin{equation*}
\begin{array}{rcl}
0 & = & (x+y)^2 - (x-y)^2 \\
0 & = & x^2 + 2xy + y^2 - x^2 + 2xy - y^2 \\
0 & = & 2xy + 2xy \\
\end{array}
\end{equation*}

That turns out to be too much by a factor of 4. So our final equation will be the above divided by 4:

\begin{equation}
\cfrac{(x+y)^2 - (x-y)^2}{4} \\
\end{equation}

See? Simple.





















\end{document}