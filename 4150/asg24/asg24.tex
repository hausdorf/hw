%Author Alex Clemmer
%CS 4150 Algorithms
%Assignment 24:
\documentclass[a4paper]{article}
\usepackage[pdftex]{graphicx}
\usepackage{fancyvrb}
\usepackage{multirow}
\usepackage{amssymb}
\usepackage{amsmath}
\usepackage{fullpage}
\addtolength{\oddsidemargin}{-.05in}
	\addtolength{\evensidemargin}{-.05in}
	\addtolength{\textwidth}{.25in}

	\addtolength{\textheight}{.25in}
\begin{document}


\section*{Assignment 24}
Alex Clemmer\\
Student number: u0458675

\section{}

The expression $\texttt{z = (z * x) mod N}$ occurs in $O(n^2)$. Since $N$ is an $n$-bit number, and since $x$ and $y$ are strictly less than $N$, the worst possible case is that $x$ and $y$ both have a value of $2^n-2$.

This means that the $O(n^2)$ statement, which easily dominates the inner part of the loop, gets iterated $2^n-2$ times. That gives us:

\begin{equation}
O(2^n \cdot n^2)
\end{equation}

Gross!

\section{}

So $\texttt{z = x * y}$ should be $O(n^2)$, depending on the algorithm you use. $N$ is an $n$-bit number, which means its worst case value is $2^n-1$, which means that the loop ends up iterating $2^n$ times asymptotically. The statement inside the loop is only slightly trickier: $\texttt{z-N}$ is linear, and allocating the result into $\text{z}$ is also linear. So either way, the loop takes $O(2^n)$, and the loop itself should take $O(n)$ time. So that gives us $O(n^2) + O(n \cdot 2^n)$, which is really just:

\begin{equation}
O(n \cdot 2^n)
\end{equation}

Better, but still horrible.

















\end{document}