%Author Alex Clemmer
%CS 4150 Algorithms
%Assignment 42:
\documentclass[a4paper]{article}
\usepackage[pdftex]{graphicx}
\usepackage{fancyvrb}
\usepackage{multirow}
\usepackage{amssymb}
\usepackage{amsmath}
\usepackage{fullpage}
\addtolength{\oddsidemargin}{-.05in}
	\addtolength{\evensidemargin}{-.05in}
	\addtolength{\textwidth}{.25in}

	\addtolength{\textheight}{.25in}
\begin{document}


\section*{Assignment 42}
Alex Clemmer\\
Student number: u0458675

\section{}

Say we have some search problem $\Pi$ with input size $n$. Since every search problem in NP can be expressed as a series of decision problems, the worst case is that we have have to examine both possible answers (``yes" and ``no") for every single input $\in [0,n]$. So there are somewhere around $2^n$ subproblems.

If it's nondeterministically solvable in polynomial time (that is, given a machine that can explore an arbitrary number of solutions simultaneously---the definition of NP), all these subproblems themselves must be also solvable in polynomial time, let's say specifically $p(n)$.

So each problem costs at most $p(n)$ for some polynomial function $p$. If there are $2^n$ subproblems at worst, then the total cost for all problems in NP must be $O(2^{p(n)})$.


















\end{document}