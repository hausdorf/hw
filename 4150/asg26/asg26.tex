%Author Alex Clemmer
%CS 4150 Algorithms
%Assignment 26:
\documentclass[a4paper]{article}
\usepackage[pdftex]{graphicx}
\usepackage{fancyvrb}
\usepackage{multirow}
\usepackage{amssymb}
\usepackage{amsmath}
\usepackage{fullpage}
\addtolength{\oddsidemargin}{-.05in}
	\addtolength{\evensidemargin}{-.05in}
	\addtolength{\textwidth}{.25in}

	\addtolength{\textheight}{.25in}
\begin{document}


\section*{Assignment 26}
Alex Clemmer\\
Student number: u0458675

\section{}

First find $\phi(N)$: $(p-1)(q-1) = 504$. This is the Euler Totient Function; we'll call it $n$ for short. Given that $e = 5$. The smallest $x$ such that $5x \equiv 1 (\mbox{mod } 504)$ is $x=101$. Thus $\mathbf{d = 101}$. Pretty simple.

\section{}

Encrypting 55 is pretty straightforward. Given $m^e (\mbox{mod } n)$, we have $55^5 (\mbox{mod } 504) = 55$. If that doesn't seem right, we can reverse it given $m^d  (\mbox{mod } 504)$: $55^{101}  (\mbox{mod } 504) = 55$.

\section{}

Given the above, this is super simple. $189^101 (\mbox{mod } 504) = 189$. We can confirm this odd result by $189^{5}  (\mbox{mod } 504) = 189$.

















\end{document}