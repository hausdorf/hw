%Author Alex Clemmer
%CS 4150 Algorithms
%Assignment 29:
\documentclass[a4paper]{article}
\usepackage[pdftex]{graphicx}
\usepackage{fancyvrb}
\usepackage{multirow}
\usepackage{amssymb}
\usepackage{amsmath}
\usepackage{fullpage}
\addtolength{\oddsidemargin}{-.05in}
	\addtolength{\evensidemargin}{-.05in}
	\addtolength{\textwidth}{.25in}

	\addtolength{\textheight}{.25in}
\begin{document}


\section*{Assignment 29}
Alex Clemmer\\
Student number: u0458675

\section{}

Transforming $x^p \equiv x (\mbox{mod }p)$ into a more familiar format is really simple. We start with prime identity given by the book, $x^{p-1} \equiv 1(\mbox{mod } p)$. Multiplying both sides by $x$, we get $x^p \equiv x(\mbox{mod } p)$. We know that we can do this because $x$ is obviously relatively prime to $x$.

\section{}

Transforming some $x^{k(p-1)+1} \equiv x (\mbox{mod } p)$ also begins with the familiar identity that $x^{p-1} \equiv 1(\mbox{mod } p)$. By raising both sides to the power of $k$ we get $x^{k(p-1)} \equiv 1(\mbox{mod } p)$. Remember that this is intuitively satisfying because we mod intuitively at every point we multiply a number by itself when raising it to some power. So if a number is congruent $\mbox{mod } p$, it should be congruent for all powers $k$ that it is raised to.

There's no black magic in the last step: we simply multiply both sides by $x$. This gives us our final solution, $x^{k(p-1)+1} \equiv x (\mbox{mod } p)$.

\section{}

$e$ and $p-1$ are relatively prime, so we can use Extended Euclid's algorithm to give $fe + b(p-1) = 1$. Mod-ing both sides by $p-1$ gives us $fe \equiv 1(\mbox{mod }p-1)$.

\section{}

For some message $y$ in $y \equiv x^e (\mbox{mod } p)$: we've found $f$ in the previous part, so our equivalence relation becomes $y^f \equiv x^{ef} (\mbox{mod }p)$. The totient function $\phi(p) = p-1$, so by Euler's we can $y^f \equiv x^{ef (\mbox{mod }p-1)} (\mbox{mod } p)$, which is then $y^f \equiv x(\mbox{mod } p)$.

Decrypting is simple: find $f$ then use $x \equiv y^f (\mbox{mod } p)$.
















\end{document}