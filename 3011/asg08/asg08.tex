%Author Alex Clemmer
%CS 3011 Algorithms
%Assignment 08:
\documentclass[a4paper]{article}
\usepackage[pdftex]{graphicx}
\usepackage{fancyvrb}
\usepackage{multirow}
\usepackage{amssymb}
\usepackage{amsmath}
\usepackage{fullpage}
\addtolength{\oddsidemargin}{-.05in}
	\addtolength{\evensidemargin}{-.05in}
	\addtolength{\textwidth}{.25in}

	\addtolength{\textheight}{.25in}
\begin{document}

\section*{Assignment 08}
Alex Clemmer \\
Student number: u0458675

\subsubsection*{Calvin Gaisford, Boyd Timothy---Appigo}

Mobile is one of the more interesting spaces in CS right now. More than 65 million people in developing countries get a smartphone in a given year, and for almost all of them, that smartphone is their only access to the internet.

I don't suppose I have to argue that this is transformative, but I will: consider that an economic crisis in 1929 looks very different than an economic crisis in 2010. Those days, people really had to rely on newspapers, if they read at all. These days, you can look at economists talk about things on YouTube, or you can read blogs of academics, or you can read papers about the economy. And perhaps best yet, the reporters who do the investigative legwork also have access to this information also, which makes them more effective. People these days are informed, or have the power to be informed, which makes companies and the government more accountable. This is only bolstered by the fact that the people inside these companies and the government are also better informed themselves. Information breathability is important.

One thing that was true for me before this talk was that I was interested to see where mobile went, but not at all impressed with what is actually done with mobile. It seems to me that companies and people that hire developers to build mobile apps tend to not be looking for people to ``just code it up" which as we all know is frustrating and lame. Mobile also seems to attract the worst sort of developers--people who think it's easy and are looking basically to just get rich.

Another thing that bothered me about mobile is that the amount of stuff you can do seems to be really limited. How many calendar apps and music apps and RSS readers do we need? What can apps really do for us? And why should I care about an app that is basically a portable billboard. I think part of this is bolstered by the fact that I have a smartphone and don't use apps at all. They're slow, buggy, and they usually add only marginal utility to my life.

I've slowly learned that this is partly my problem. Earlier this year, for example, QuestVisual released an iPhone app\footnote[1]{http://questvisual.com/} that reads text via the iPhone camera, translates it on the fly, and then re-renders it $\textit{in place of the original text}$. If that sounds incredible, go look at the video on their site. This teaches me that apps $\textit{can}$ be cool, useful, and well-suited to their environment.

This sort of information is, I suppose, easy to glean in general, especially if you follow tech media closely. One thing that was not obvious is how hard and dedicated the really good app developers are, which is one of the things I took away from this talk. They spent months developing their first app, and they did it full time, in addition to (good, established) day jobs. They seem to have succeeded because they really tried hard to do well what they were doing.

Also sort of surprising was how fertile the ground was for hacking. The iPhone API was apparently a hodgepodge of invisible walls and inconsistencies, and building an app required a nose for hacking over these issues, which is something that not just any coder can do really well. In a way, after their description, it even seems like you really need incredible coders to pull off a really good iphone app. It occurs to me as I write this that Calvin and Boyd were probably successful at least in part because they were genuinely better coders than most of the people developing apps, and this is not least because most established coders work at jobs like the ones they had before they started this company.

It is still sort of odd to me that someone would work for a company like Appigo. The iPhone SDK is relatively cheap, xcode is free, and you don't strictly speaking $\textit{need}$ anyone else (or, really, a whole lot else at all) in order to build an app. As a hacker, it's hard for me to understand why you would dedicate so much time and effort to building something, and then just let someone else get all the money for it. Why not just go do it yourself? Why not profit yourself?

Another thing that is still sort of mystifying to me is that you can make enough money to power a multi-person company, just based on apps for iDevices. How can an app like their fuel calculator really succeed at the level it did? I guess it's hard for me to understand because I do not even understand what problem in my life these apps are supposed to fill. From my perspective it seems like people are more likely to use apps because they have a smartphone than because apps are amazing in general. I suppose, though, that there must be more to it than novelty, because I can imagine situations where something like the App Store would have failed miserably. And indeed, I think it is a compelling argument that the Android store is only remotely successful because Apple did it so successfully first. (Consider that a typical iPhone app will still make about 3 times as much as a typical Android app on average.)

Actually, thinking about it further, the other reason it is mystifying to me that any App company is successful at all is because it's really hard to discern what the future of Apps will look like. Surely the App Store can't continue on the trajectory it does. There's only so much screen real estate and compute power and memory--surely we'll run out of cool things to do.

But in another sense, these questions make mobile more intriguing. After all, people said these things about the Internet, too. And that brings up an important point: it's not actually the app technology itself that's really important here--it's the connectivity. The most popular apps overwhelmingly are interfaces to the Internet. They look up restaurants, sync your Facebook, and calculate your location. That's a significant advantage that apps have over mobile browsers: they're just $\textit{better suited}$ to mobile resources than JavaScript or HTML. They are better interfaces because they are faster, more robust, and they make better use of the screen.

The main drawback to this is that there is a significant decrease in flexibility. You can't just surf to a different page. Apps are not an open window to the world in the way that browsers are. This doesn't seem like a problem until you consider that a large reason that Facebook and Google are successful is that they lead you offsite on a regular basis. They are gateways to the massive world that is the Internet. Before we all jump on the app bandwagon, we should remember why it is that they're successful, and what it is that made the Internet successful to begin with. These are important questions.

























\end{document}