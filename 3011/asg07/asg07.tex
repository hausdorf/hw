%Author Alex Clemmer
%CS 3011 Algorithms
%Assignment 07:
\documentclass[a4paper]{article}
\usepackage[pdftex]{graphicx}
\usepackage{fancyvrb}
\usepackage{multirow}
\usepackage{amssymb}
\usepackage{amsmath}
\usepackage{fullpage}
\addtolength{\oddsidemargin}{-.05in}
	\addtolength{\evensidemargin}{-.05in}
	\addtolength{\textwidth}{.25in}

	\addtolength{\textheight}{.25in}
\begin{document}

\section*{Assignment 07}
Alex Clemmer\\
Student number: u0458675

\subsubsection*{Kris Johnson and Clark Stacey---From AI to Z-Buffers: Programming Disciplines of the Videogame Industry}

Making games is clearly one of the stranger release cycles in CS: you make game, Kris and Clark tell us, and then you ship it, and then that's it. You spend anywhere from a year to 5 years developing it, and 6 months after release, no one else buys your product ever again. There are no updates. And enough people need to buy your game in this time to make a profit.

The actual discipline of design is a bit different to: most things in games really only have to work approximately. Where the AI needs to be really precise at Google, they only have to be $\textit{not obviously wrong}$ in a game, and of course the same goes for collision detection, pathfinding, and so on. The downside of this laxness is that you really do need to make sure the game is over the fuzzy correctness threshold, because there are no software updates. 

This is the most striking thing about the development process to me. In other subareas of CS, there is at least some failure tolerance. In game development there is almost none. This, I suppose, leads to an incredible focus and dependence on amazing tools. They have tools for modeling, rendering, development pipelining, and so on.

This is something I've heard quite a lot, especially from people at larger companies. Yishan Wong, for example, (the executive arguably most responsible for the engineering culture at Facebook) claims that development of tools is literally more important than development of features $\footnotemark$.

Interestingly, it seems to be only management who bring this up. When engineers speak ($\textit{e.g.}$, the Backcountry presentation), they tend to focus purely on engineering problems. But it's interesting, and in some ways necessary, to know how and why the company you work for is successful. Or at least it's useful for engineers who are potentially entering the job market soon to know about the game they're getting into.

One constant mystery for me is how industry management sorts through the massive amounts of information to pick a direction for the company. There are a lot of anecdotal stories that can lead you along the way (Kris told an awesome story about the rise of World of Warcraft in China), but anecdotes alone are no basis for decisions that involve millions of dollars. Even sorting through the information to find what's relevant is intimidating.

From the engineering student's perspective, it was most useful to hear what the company looks for in employees. They mentioned some things we knew, -- that personal projects are pretty much required in industry -- as well as some things we (or I, anyway) didn't -- that the EAE program is very influential in vetting internship candidates. This is useful even for people (like me) who are not at all interested in game development.

Also useful was the insight into the (complicated) dev pipeline. Video games are complicated, large, and interesting engineering projects, and on top of that, they need to be completed really fast. Mobilizing both a team of artists and a team of engineers, and getting them to work together probably represents a significant communication challenge.

For the EAE people in the room, I imagine that the discussion of rendering, modeling, and programming games was what took the cake. For me it was the discussion of the future of the gaming, and this was mainly because the future of gaming is directly related to the sorts of devices people will use in the future. Both Kris and Clark seem to believe strongly that the future is in mobile.

In particular, the idea that \$1 billion worth of Android devices will be sold in the next year, a lot of them in developing countries, and to people who otherwise do not have internet access. This, of course, is not a factor that is only relevant to games; 100 million people in developing countries coming online in a given year is absolutely disruptive. It's a sea change in the sorts of things that are possible in those countries, and it is not only video game companies who should pay attention.

$\footnotetext[1]{http://algeri-wong.com/yishan/engineering-management-tools-are-top-priority.html}$

























\end{document}