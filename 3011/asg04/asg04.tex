%Author Alex Clemmer
%CS 3011 Algorithms
%Assignment 04:
\documentclass[a4paper]{article}
\usepackage[pdftex]{graphicx}
\usepackage{fancyvrb}
\usepackage{multirow}
\usepackage{amssymb}
\usepackage{amsmath}
\usepackage{fullpage}
\addtolength{\oddsidemargin}{-.05in}
	\addtolength{\evensidemargin}{-.05in}
	\addtolength{\textwidth}{.25in}

	\addtolength{\textheight}{.25in}
\begin{document}

\section*{Assignment 04}
Alex Clemmer\\
February 22, 2011\\
Student number: u0458675

\subsubsection*{John Regehr -- Is Grad School Right for You?}

John's talk was half an all-purpose things-you-need-to-know information session about grad school, and half a discussion about the difficulties he disliked having to discover on his own when he was in grad school.

John of course made a lot of the obligatory points that people seem to always be surprised by. I think in a lot of ways educators are bound to report that grad school is not always (ever?) the avatar that students seem to think it is, and that's probably because people in general seem to  $\textit{think}$ that they understand a lot more about the world than they really do, which usually leads to bad, or at least uninformed, decisions.

One thing that should strike students in general -- if I understand what John was saying -- is that grad school is not really like ``school" at all, or not as students traditionally think of it. Somewhere along the way, students seem to be expected to transfer from knowledge consumers to knowledge producers, and I bet that catches most people off guard, including people who knew it was coming.

The interesting thing is that this is not a secret; $\textit{every}$ professor seems to say this, so in theory this surprise should be no surprise at all. The divide seems to occur where people are told $\textit{about}$ research and jobs and the like: when students are told how grad school is different, they will process it typically through the lens of the traditional school experience. Professors, in contrast, learned why grad school is different by actually going to grad school; it would be unrealistic to expect that their students are somehow above this.

The impact of this is that, in some ways, John seemed to be constrained by the fact that he probably needed to address some number of topics for the talk to be acceptable. I think it would be easy to turn a talk like this into a half-hearted list of bullet points of things in grad school that people ``need to know". I imagine people are less apt to pay attention to something like this, and more importantly, they are probably more apt to assume they understand where they do not understand at all.

John, I think, took the right approach, which was to frame the talk as an informal discussion about the realistic costs and benefits of each particular choice one could make for their post-grad life. This seemed to give him the flexibility to discern exactly what content was relevant to the audience, which among other things meant that people who have done a lot of research about grad school (like me) ended up learning things too.

One thing I found particularly useful was the treatment of what one does after a PhD. Talks about grad school traditionally stop immediately after covering what makes a grad student successful; questions about things like research labs or professorships, and what makes one candidate preferable to another in academia was, and is, something that I have a hard time finding. This alone made the talk very valuable to me.

Another thing I appreciated was the candor of the talk. One question someone asked was whether it was better to go to grad school here or somewhere else. I know more than a few faculty who would encourage students to get their graduate degree here, either through the 5-year master's program, or the PhD program. John, in contrast, said that the unequivocally best decision was almost certainly to go somewhere else, just for the intellectual ventilation.

For most of the rest of the students in the class, the takeaway message seemed to be to get your master's, and then quit. John argues that this is the ``optimal" move if you live in the ``silly" world where your only goal is to make the most money. This is probably true, and this is the part most students will probably remember. Less memorable, however, is that not everyone is suited to grad school, and if your end goal is to be really good at what you're doing, this may not actually be the best thing to do.

Even though some of the more important things probably got lost in the ether, ultimately I think it's valuable to force this on people, because most students don't spend an extended amount of time talking to professors about grad school, even though they should. One thing that John did do well was to directly challenge what students (at least in my experience) tend to say about grad school, which is something that they hopefully at least give some cursory thought to as they decide which path to take later.

That said, I do think it was possible to take what you wanted to hear away from the talk, rather than what was meant. In that case, though, you're probably not really interested in discerning the truth anyway, and although it's a significant enough problem (I bet that actually $\textit{most}$ students fall into this category), I'm not certain that it can be corrected at all. A teacher's responsibility only extends so far.

























\end{document}