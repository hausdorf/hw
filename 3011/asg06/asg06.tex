%Author Alex Clemmer
%CS 3011 Algorithms
%Assignment 06:
\documentclass[a4paper]{article}
\usepackage[pdftex]{graphicx}
\usepackage{fancyvrb}
\usepackage{multirow}
\usepackage{amssymb}
\usepackage{amsmath}
\usepackage{fullpage}
\addtolength{\oddsidemargin}{-.05in}
	\addtolength{\evensidemargin}{-.05in}
	\addtolength{\textwidth}{.25in}

	\addtolength{\textheight}{.25in}
\begin{document}

\section*{Assignment 06}
Alex Clemmer\\
March 13, 2011\\
Student number: u0458675

\subsubsection*{Larry Seppi -- Rockwell Collins Simulation and Training Solutions}

A lot of Larry's presentation centered around the process of product development in a large, very corporate firm. This turned out to be an invaluable presentation because I am pretty sure I would not survive in an environment like that.

In the first place, Larry explicitly confirmed what I have feared: life at Rockwell Collins (and, he says, at similar firms, like IBM) tends to be regimented and disciplined. People come to work at 8 and leave at 5. The freedom to pursue something that you think will help the company on your own does not exist as it would at ``trendy" firms that seem to be prevalent today. You have to clear days off with HR and working overtime is not really smiled upon.

I do wonder how firms like RC actually attract talent when there are so many firms like Backcoutry out there. I can tell you personally who I'd rather work for, and I can tell you who all the really smart people I know would want to work for. L3 seems to recruit people in college by helping to pay for it, but this doesn't necessarily get you the people who are most passionate, about doing amazing things. I think Larry did try to address this question (I asked it during the lecture), but I didn't get the impression that he understood in great detail why or how people ended up at RC. I'm excited to hear L3's answer when they come through.

It was very interesting to hear about their product cycle. A lot of the process seems to be informed by the fact that the cost of failure in their market is really high---just getting involved in the field requires a massive investment in hardware, and then on top of that there's the investment required for the software. I get the impression that this too is pretty expensive by virtue of the fact that their software is tightly coupled with their hardware development. I guess it's sort of like $O()$ notation, because the dominant-cost ends up dominating the lifecycle of the cheaper components.

Another thing that's really interesting is how different the role of management is in RC versus other firms. When Backcountry spoke, Shane Hansen and Ross Solomon mentioned that their managers tend to spend most of the day coding, where a manager at RC would spend most of her day in meetings, managing costs, or figuring out corporate trajectory. I highly doubt that there $\textit{none}$ of that happens at firms like Backcountry, but it's pretty obvious that management means something different between the two.

One problem that this approach seems to emphasize is the need of fundamental transformations at every level of promotion. For example, an engineer at RC will tend to do mostly engineer-y stuff. But when they're promoted to manager, they do mostly corporate meetings and things like that. Then when they get promoted again, they have to manage many managers, which again involved a fundamental transformation of the job.

I read somewhere that a major problem in management is that the things that make a person good at being an engineer $\textit{usually}$ end up making them bad at being a manager. I'm pretty sure this applies mostly to the more ``classic" model of management you would find at RC, and I'm interested to see how they go about finding managers. It must be hard to find good engineers who want to make that transition. The very best engineers I know want to be engineers almost exclusively.

Something that is striking about RC in particular is that it seems to not have had to change much over the years. Most of their employees have been there for several years, if not several decades. And in a lot of ways that seems to give a good intuitive explanation of the culture: why change to something more trendy when none of your employees particularly want to see that happen? But RC actually does need to hire younger employees, not just for the usual reasons, but also for self-preservation: someday the older people will retire, and then someone else will need to run the company. I suppose they could hire outside, but en-masse that has been historically disastrous.

That transition is very interesting to me. Companies have made the transition before, but it seems to depend a lot on luck. It will be interesting to see how RC deals with this, because ultimately what they do as a company and what options they have for growth will depend on the intellectual real estate that is their employee pool. Hopefully they pull it off.

























\end{document}