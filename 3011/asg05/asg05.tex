%Author Alex Clemmer
%CS 3011 Algorithms
%Assignment 05:
\documentclass[a4paper]{article}
\usepackage[pdftex]{graphicx}
\usepackage{fancyvrb}
\usepackage{multirow}
\usepackage{amssymb}
\usepackage{amsmath}
\usepackage{fullpage}
\addtolength{\oddsidemargin}{-.05in}
	\addtolength{\evensidemargin}{-.05in}
	\addtolength{\textwidth}{.25in}

	\addtolength{\textheight}{.25in}
\begin{document}

\section*{Assignment 05}
Alex Clemmer\\
March 4, 2011\\
Student number: u0458675

\subsubsection*{Shane Hansen, Ross Solomon -- Distributed Systems in Practice}

Shane and Ross's presentation was very helpful. Spending years in academia without having actually gone into industry leads to all sorts of weird conceptions about how the ``real" world works. It's generally extremely helpful to talk to people who had $\textit{exactly}$ the same questions as you a couple of years ago. Shane and Ross turned out to be exactly those people, and Backcountry (as far as I can tell) tends to fit the typical tech company mythos.

There were a number of things that were interesting here (at least to me). The first was that the engineers had a pretty intimate understanding of the scaling issues the company went through. One thing that was sort of surprising to me was that there was no ``Reddit bump" (or whatever the equivalent was in their time)---Backcountry's popularity seems to have come slowly, and apparently largely because of persistence. So rather than having to contend with serving lots of pages, they have to contend with database scaling issues, which is not at all the first thing one thinks of when one thinks of scaling issues.

Another reason this is sort of interesting is the fact that Backcountry is transitioning away from older (Perl-based) technology, which is something that many companies simply never get around to doing, or do poorly enough that it ruins them. This agility is in a lot of ways what people expect from CS companies these days, but it is by no means easy to pull off, and I'm sort of impressed that they seem to be successful at it. This is even more obvious as most of the other companies we've seen are very much $\textit{not}$ this agile. And especially not companies as old as Backcountry.

This is especially impressive when you consider how traditional companies of a similar age are. I was talking to a guy from ebay at the career fair, and he was talking about the (incredibly corporate) product cycles they go through, and how they use things like $\texttt{svn}$ as opposed to the  mercurial or $\texttt{git}$, mostly just because that's the way they have always done it. Backcountry was started in the 90's too, but it seems to be in a constant state of flux, where the employees are allowed to choose the tools that are best for the job, rather than to bend to a corporate dictum just because that's the formality. I hope to work in an environment like that if I go into industry, and I think most people feel the same.

A lot of the rest of Ross and Shane's presentation was built around explaining what they (and other employers in general) expect from candidate employees. One thing they went out of their way to say is a point that I frankly don't think is made enough---that employers want to see quantitative projects that students have done on their own. Most computer science students do not seem to be aware that almost all employers more or less expect this, and although I expect almost all of them will ignore this, it's probably very important to recognize that most people who are really good are also really passionate, and thus want to work mainly with other passionate people.

One thing that I became very curious about as time went on was the exact process of determining company direction. I notice, for example, is that even the engineers know that feature dev is frozen during the holiday season in favor of site consistency, which indicates that someone sat down and thought a lot about how the company should work, and what patterns were relevant to their central objective. I would have liked to hear a lot more about how they sift through the mountains of data to find not just their main objectives, but what the best way to accomplish them is. From the perspective of an outsider, it seems almost impossible.

The other thing I would like to have heard a lot more about was the structure of a company like Backcountry. Is the management mostly business people, or are they engineers who graduated upwards? I'm very committed to finding a job that's a good fit for me, and I do wonder whether that's engineering or some sort of management position, and it would have been nice to have a treatment of the sorts of people who are happy doing both.

Yet another thing that I'm very interested in generally speaking is the methods a company like Backcountry uses to acquire great talent. It's fairly obvious what employers look for, but how employers compete for those employees is much less obvious, and a lot more interesting, at least to me, especially since they do have to compete with a lot of other great places to work. This is something that every company does differently, and Backcountry (at least according to my reading) tends to be good at it.

























\end{document}