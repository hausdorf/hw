%Author Alex Clemmer
%CS 3011 Algorithms
%Assignment 02:
\documentclass[a4paper]{article}
\usepackage[pdftex]{graphicx}
\usepackage{fancyvrb}
\usepackage{multirow}
\usepackage{amssymb}
\usepackage{amsmath}
\usepackage{fullpage}
\addtolength{\oddsidemargin}{-.05in}
	\addtolength{\evensidemargin}{-.05in}
	\addtolength{\textwidth}{.25in}

	\addtolength{\textheight}{.25in}
\begin{document}

\section*{Assignment 02}
Alex Clemmer\\
January 21, 2011\\
Student number: u0458675

\subsubsection*{Joe Sindad -- Career Planning}

Mr Sindad spent most of the talk giving practical advice about the information sieve that students tend to encounter in the transition from school to whatever it is that comes after school. Mostly this consisted of advice about resumes and job interviews, but also a lot of it had to do with making students aware of the resources that the U makes available to them ($\textit{i.e.}$, in the form of practice interviews).

Being good at marketing oneself in general is a really excellent trait to have, but in general I find the basic questions about how to ($\textit{e.g.}$) build a CV to be useful only to a point. I have a young man's opinion, but in my experience, when you spend a lot of time actually trying to do really awesome things, your resume just sort of falls into place. That's especially true if you have a really awesome advisor who can look over it and tell you what you need to omit, or what doesn't work. I wish that this point would get more press.

Another things is, I notice that a lot of talks about how to write a resume tend to (sometimes implicitly) place the focus on $\textit{writing}$ the resume, which may not be the most important thing to think about. I'm not saying that it's not important how you write it or what you put on it, but most mistakes can be caught by a good once-over from your advisor, or by a little thought about what would be important to you if you were hiring (and failing that, research about what other people actually do look for).

In other words, to me, it seems like a much more important part of building a resume is to spend a lot of time thinking about what your interests are professionally, and then to have done things independently and by your own initiative that have given you experience solving the sorts of problems you would end up solving in your field. It also obviously helps to have had a good network of people around you to advise your direction and abet your progress (and hopefully theirs by extension). I think that mistakes in formatting a resume tend to be a lot less important than having done nothing interesting. This might sound obvious, and I don't want to belabor the point, but when I was in a place where this presentation would have helped me a lot, I feel that I would have been better served by a statement something akin to what I have just described.

Of course, I don't mean to be extraordinarily critical here, but these are the things that I honestly did think about listening to the talk. I could definitely be wrong, but in my opinion it is much more important to critically analyze evidence at your disposal and extend your view as new evidence requires than it is to be agreeable all the time. Hopefully this is a value we share.

On that note, another thing off the top of my head that I noticed was that the model of interview presented seems to be a bit out of date, or at the least, too simplified. I felt like interviews at CS companies were presented as a series of interesting questions about polar bears and manhole covers, and it seems true that this is the hallmark of a tech interview, but I think it is more important (and more illustrative) to try and understand $\textit{why}$ tech interviews are so often a battery of subtlety and rigor.

One reason seems to be that successful companies $\textit{need}$ to hire the best people to succeed. This seems obvious, but it is important for different reasons than one would think: in computer science people capital is probably the only thing that really matters. Consider that if I want to build and deploy some software product (think Dropbox), I often need little more than command line, a text editor, and my toolchain. I can name dozens of successful companies that never took venture capital at all (like Github). In contrast, a biotech company would need millions in equipment alone to get started.

The major impact here seems to be that people capital matter disproportionately more in CS than they do in other industries, which are slower due to the fact that there must be more overhead to manage the scarcity of resources---we don't need to take turns to use the only computer at the company. I think that this is the reason companies like Google can afford to pay people to do almost whatever they want: when failure is so absurdly cheap $\textit{and}$ you hire the best, you win.

Another thing that seems to be relevant is that companies that are at all worth working for tend to make hiring the absolute first priority. The gold standard here is Google and Facebook, and in both cases, an intimate involvement with the hiring process (interviewing, hiring decisions, et al) are absolutely mandatory for everyone, which first and foremost means that the engineers who ``just want to code" tend to be eliminated in favor of the people who are excited to find and work with the very, very best. Another important impact is that it is no longer the manager's job alone to make sure that everyone is up to snuff, and after a certain critical mass, the culture becomes self-sustaining.

Of course, one might think that this is less important for smaller companies, but I don't see any evidence that this is the case. Even Mark Polson affirmed this notion personally to me last week, and of course, I've heard similar things in blogs and interviews. And intuitively, given the fierceness of the competition, I would imagine that it would be $\textit{more}$ important to be really good at a small scale, where one person can literally ruin the company with one or two bad decisions.

There are obviously more reasons, but we have enough, I think, to make a basic conclusion, which is that probably much more important than whether you can answer absurd questions is what sort of person you are. And whether you agree with what I've said or not, it can't be a good thing to ignore the fact that when circulating your resume, you should give serious, substantive, and careful though to $\textit{where you actually belong}$. You shouldn't work at Google because it's great, you should work there because it's the right place for you.

In the end, I wouldn't say it was a $\textit{bad}$ presentation, but I felt like there were some pretty big omissions, and I feel like it may not have benefited people who ($\textit{e.g.}$) have already built a resume and gotten internships. The things that I felt were most valuable tended to be specialized information that the speaker had specialized access to, $\textit{i.e.}$ the resources the U makes available to students transitioning into the job market. I actually did feel that was valuable to know.

\end{document}