%Author Alex Clemmer
%CS 3130 Eng Stats Prob
%Assignment 6:
\documentclass[a4paper]{article}
\usepackage[pdftex]{graphicx}
\usepackage{fancyvrb}
\usepackage{multirow}
\usepackage{amssymb}
\usepackage{amsmath}
\usepackage{haldefs}
\usepackage{fullpage}
\addtolength{\oddsidemargin}{-.05in}
	\addtolength{\evensidemargin}{-.05in}
	\addtolength{\textwidth}{.25in}

	\addtolength{\textheight}{.25in}

\begin{document}

\section*{Assignment 6}
Alex Clemmer\\
Student number: u0458675

\subsection*{Problem 1:} 

\paragraph{(a)} 

\renewcommand\arraystretch{1.5}
\begin{tabular}{ c | c | c | c | c | c | c }
& \multicolumn{4}{c}{$R$} \\
\hline
\multirow{7}{*}{$D$} & & 0 & 1 & 2 & 3 & 4 \\
\cline{2-7}
& 0 & & & & & $\frac{188}{3201}$ \\
\cline{2-7}
& 1 & & & & $4 \cdot \frac{200}{3201}$ & \\
\cline{2-7}
& 2 & & & $6 \cdot \frac{1225}{19206}$ & & \\
\cline{2-7}
& 3 & & $4 \cdot \frac{200}{3201}$ & & & \\
\cline{2-7}
& 4 & $\frac{188}{3201}$ & & & & \\
\end{tabular}
 \\\\\\
Note that all the both the marginal distributions and the whole table add to 1, although I didn't draw that into the table. I figured this out by building this huge tree and putting the values together that had the same $\textit{type}$ of composition ($\textit{e.g.}$, three of a kind, all are one kind, etc.).

\paragraph{(b)}

First, we need to find the $\Ep[RD]$:

\begin{equation}
\begin{array}{rll}
\Ep[RD] & = & (4 \cdot 0) \left( \cfrac{188}{3201} \right) + (3 \cdot 1) \left(4 \cdot \cfrac{200}{3201} \right) + (2 \cdot 2) \left(6 \cdot \cfrac{1225}{19206} \right) + (1 \cdot 3) \left(4 \cdot \cfrac{200}{3201} \right) + (0 \cdot 4) \left( \cfrac{188}{3201} \right) \\[.05in]
& = & \cfrac{100}{33} \\[.05in]
\end{array}
\end{equation}

We will also need to multiply $\Ep[R]$ and $\Ep[D]$ together, so we find them next:

\begin{equation}
\begin{array}{rll}
\Ep[R] & = & 1\left(4 \cdot \cfrac{200}{3201} \right) + 2\left(6 \cdot \cfrac{1225}{19206} \right) + \left(4 \cdot \cfrac{200}{3201} \right) + \left( 4 \cdot \cfrac{188}{3201} \right) \\[.05in]
& = & 2 \\[.05in]
\end{array}
\end{equation}

The same holds for $\Ep[D]$, as they are symmetric. Now we put them all together:

\begin{equation}
\begin{array}{rll}
\mbox{Cov}(R, D) & = & \Ep[RD]-\Ep[R] \Ep[D] \\[.05in]
& = & \cfrac{100}{33} - (2 \cdot 2) \\[.05in]
& = & -\cfrac{1}{162} \\[.05in]
\end{array}
\end{equation}


\paragraph{(c)}

First we need to find both Var($R$) and Var($D$):

\begin{equation}
\begin{array}{rll}
\mbox{Var}(R) & = & 1^2 \left(4 \cdot \cfrac{200}{3201} \right) + 2^2 \left(6 \cdot \cfrac{1225}{19206} \right) + 3^2 \left(4 \cdot \cfrac{200}{3201} \right) + 4^2 \left( 4 \cdot \cfrac{188}{3201} \right)  \\[.1in]
& = & \cfrac{164}{33} -4 \\[.1in]
& = & \cfrac{32}{33} \\[.05in]
\end{array}
\end{equation}

Var($D$) will be the same. After that, it's easy to plug them into the equation:

\begin{equation}
\begin{array}{rll}
\rho(R,D) & = & \cfrac{\mbox{Cov}(R,D)}{\sqrt{\mbox{Var}(R)\mbox{Var}(D)}} \\[.15in]
& = & \cfrac{-\cfrac{32}{33}}{\sqrt{\cfrac{32}{33} \cdot \cfrac{32}{33}}} \\[.2in]
& = & -1\\[.1in]
\end{array}
\end{equation}


\subsection*{Problem 2:}

\paragraph{(a)}

First we eliminate $y$ from the equation:

\begin{equation}
\begin{array}{rll}
F_X(x) & = & \displaystyle\int^1_{0} \frac{2}{3}(x+2y) \,dx \\[.15in]
& = & \left[ \cfrac{2xy+2y^2}{3} \right]^1_0 \\[.15in]
& = & \cfrac{2}{3}(x+1) \\[.15in]
\end{array}
\end{equation}

Now we can integrate the rest of the equation to find $P(\frac{1}{2} \le X \le 1)$:

\begin{equation}
\begin{array}{rll}
P(\frac{1}{2} \le X \le 1) & = & \displaystyle\int^1_{\frac{1}{2}}F_X(x) \,dx \\[.15in]
& = & \displaystyle\int^1_{\frac{1}{2}} \cfrac{2}{3}(x+1) \,dx \\[.15in]
& = & \left[ \cfrac{x^2}{3}+ \cfrac{2}{3} \right]^1_{\frac{1}{2}} \\[.15in]
& = & \cfrac{7}{12} \\[.15in]
\end{array}
\end{equation}

\paragraph{(b)}

First, let's find $\Ep[XY]$:

\begin{equation}
\begin{array}{rll}
\Ep[XY] & = & \displaystyle\int^1_{0} \int^1_{0} x \cdot y \frac{2}{3}(x+2y) \,dy  \,dx \\[.15in]
& = & \displaystyle\int^1_{0} \left[ \frac{x^2 y^2}{3} + \frac{4x y^3}{9} \right]^1_0 \,dx \\[.15in]
& = & \displaystyle\int^1_{0} \frac{x^2}{3} + \frac{4x}{9} \,dx \\[.15in]
& = & \left[ \cfrac{x^3}{9} + \cfrac{2x^2}{9} \right]^1_0 \\[.15in]
& = & \cfrac{1}{3} \\[.15in]
\end{array}
\end{equation}

Next, we find $\Ep[X]\Ep[Y]$. We start with the marginal pdfs found earlier:

\begin{equation}
\begin{array}{rll}
\Ep[X] & = & \displaystyle\int^1_{0} x f_X(x) \,dx \\[.15in]
& = & \displaystyle\int^1_{0} x \cfrac{2x+2}{3} \,dx \\[.15in]
& = & \left[ \cfrac{2x^3}{9} + \cfrac{x^2}{3} \right]^1_0 \\[.15in]
& = & \cfrac{5}{9} \\[.15in]
\end{array}
\end{equation}

\begin{equation}
\begin{array}{rll}
\Ep[Y] & = & \displaystyle\int^1_{0} y f_Y(y) \,dy \\[.15in]
& = & \displaystyle\int^1_{0} y \cfrac{4y+1}{3} \,dy \\[.15in]
& = & \left[ \cfrac{4y^2}{9} + \cfrac{y^2}{6} \right]^1_0 \\[.15in]
& = & \cfrac{11}{18} \\[.15in]
\end{array}
\end{equation}

Now we can put it all together:

\begin{equation}
\begin{array}{rll}
\Ep[XY] - \Ep[X]\Ep[Y] & = & \cfrac{1}{3} - \left( \cfrac{5}{9} \right) \left( \cfrac{11}{18} \right) \\[.15in]
& = & \cfrac{-1}{162} \\[.15in]
\end{array}
\end{equation}

\paragraph{(c)}

Variance is still $\Ep[X^2]-\Ep[X]^2$. We are going to find the variance of both RVs before anything else. We can build off of what we've determined in the previous problems.

The following is for $X$:

\begin{equation}
\begin{array}{rll}
\Ep[X^2] & = & \displaystyle\int^1_{0} x^2 f_X(x) \,dx \\[.15in]
& = & \displaystyle\int^1_{0} x^2 \cfrac{2x+2}{3} \,dx \\[.15in]
& = & \left[ \cfrac{x^4}{6} + \cfrac{2x^3}{9} \right]^1_0 \\[.15in]
& = & \cfrac{7}{18} \\[.15in]
\end{array}
\end{equation}

We can use the results from (b) to find the variance:

\begin{equation}
\begin{array}{rll}
\Ep[X^2] - \Ep[X]^2 & = & \cfrac{7}{18} - \left( \cfrac{5}{9} \right)^2 \\[.15in]
& = & \cfrac{13}{162} \\[.15in]
\end{array}
\end{equation}

We use a similar strategy to find the variance for $Y$:


\begin{equation}
\begin{array}{rll}
\Ep[Y] & = & \displaystyle\int^1_{0} y^2 f_Y(y) \,dy \\[.15in]
& = & \displaystyle\int^1_{0} y^2 \cfrac{4y+1}{3} \,dy \\[.15in]
& = & \left[ \cfrac{y^4}{3} + \cfrac{y^3}{9} \right]^1_0 \\[.15in]
& = & \cfrac{4}{9} \\[.15in]
\end{array}
\end{equation}

\begin{equation}
\begin{array}{rll}
\Ep[Y^2] - \Ep[Y]^2 & = & \cfrac{4}{9} - \left( \cfrac{11}{18} \right)^2 \\[.15in]
& = & \cfrac{23}{324} \\[.15in]
\end{array}
\end{equation}

\begin{equation}
\begin{array}{rll}
\rho(R,D) & = & \cfrac{\mbox{Cov}(R,D)}{\sqrt{\mbox{Var}(R)\mbox{Var}(D)}} \\[.2in]
& = & \cfrac{\frac{-1}{162}}{\sqrt{\frac{23}{324} \cdot \frac{13}{162}}} \\[.2in]
& = & -\sqrt{\frac{2}{299}}
\end{array}
\end{equation}

\subsection*{Problem 3:}

\paragraph{(a)}

$\Ep[X]$ and $\Ep[Y]$ will look pretty similar. They are both binomials, and it is a matter of simply substituting in one variable for another. Here we have a binomial distribution: $p(x_i) = r$, while $1-p = g+1-(r+g)$, as the number of blues is the number of marbles that are neither green nor red. This makes calculating $\Ep[X]$ $\textit{much}$ easier:

\begin{equation}
\begin{array}{rll}
\Ep[X] & = & \displaystyle\sum\limits_{n} x_n p(x_n) \\
& = & np(x_n) \\
& = & nr \\
\end{array}
\end{equation}

Similarly, $\Ep[Y] = ng$.

\paragraph{(b)}

Only slightly trickier, and also made a lot easier by it being binomial:

\begin{equation}
\begin{array}{rll}
\mbox{Var}(X) & = & np(1-p) \\
\end{array}
\end{equation}

$p$ in the case of $X$ is $r$. $(1-p)$ is not that much trickier. We know $r$ and $g$, so $(1-p) = (1-(r+g))$. So:

\begin{equation}
\begin{array}{rll}
\mbox{Var}(X) & = & nr(1-(r+g)) \\
\end{array}
\end{equation}

And:

\begin{equation}
\begin{array}{rll}
\mbox{Var}(Y) & = & ng(1-(r+g)) \\
\end{array}
\end{equation}

\paragraph{(c)}

These variables are not covariant; there is not dependence. Therefore:

\begin{equation}
\begin{array}{rll}
\mbox{Var}(X+Y) & = & \mbox{Var}(X) + \mbox{Var}(X) \\
& = & nr(1-(r+g)) + ng(1-(r+g)) \\
\end{array}
\end{equation}

\paragraph{(d)}

As said before, the covariance is 0 when the variables are independent.

\paragraph{(e)}






\end{document}